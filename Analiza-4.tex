\documentclass{article}
\usepackage{amsmath}
\usepackage{xcolor}
\title{Diferencialne enačbe}

\begin{document}
    \maketitle
    
    \section{Uvod}
    \begin{enumerate}
        \item Definiraj parcialno diferencialno enačbo
        \item Definiraj red enačbe
        {\color{red}\item Kdaj imamo eno funkcijo in kdaj sistem funkcij?}
        \item Kaj je rešitev PDE?
        \item Navedi tri oblike PDE in opiši vsako on njih
        \begin{itemize}
            \item Linearna
            {\color{red}\item Semilinearna}
            {\color{red}\item Kvazlinearna}
        \end{itemize}
        \item Navedi izreke o rešljivosti PDE
        \item Koliko rešitev je v splošnem? Kako dobimo eno samo rešitev?
        {\color{red}\item Definirajs $F$ kot vektorsko polje. Kaj velja za rešitev PDE? (zakaj?)}
        \item Opiši rešitev PDE geometrijsko
        \begin{itemize}
            \item Kak graf pripada funkciji u? Kaj je to geometrijsko?
            {\color{red}\item Kaj je definirano na tej ploskvi?}
            \item Kaj sta tangenti na krivulji v neki točki?
            {\color{red}\item Čemu je enaka linearna ogrinjača teh dveh tangent?}
        \end{itemize}
        \item Definiraj - povzami - kaj je tangentna ravnina?
        {\color{red}\item Kaj podaja vsak par funkcij $f$ in $g$?}
        {\color{red}\item Kako je polje invariantno?}
        {\color{red}\item Kakšen je predpis za polje ravnin?}
        {\color{blue}\item Definiraj integrabilno polje}
        \item Kaj je avtonomen sistem?
        \item Kaj je avtonomen sistem NDE 1. reda?
        \item Kako ga še lahko zapišemo?
        {\color{red}\item Kaj je rešitev sistema?}
        {\color{blue}\item Kaj torej velja za tangente funkcije gama?}
        {\color{red}\item Kako imenujemo funkcijo gama?}
        {\color{red}\item Kaj pove eksistenčni izrek o tej krivulji?}
        \item Velja kaj podobnega za polje ravnin? 
        {\color{blue}\item Navedi Stokesov izrek na vektorskem polju $(f(x, y), g(x, y))$}
        {\color{blue}\item Kdaj je polje potencialno?}
        \item Kaj velja na majhnih območjih?
    \end{enumerate}

    \section{Število rešitev PDE}
    \begin{enumerate}
        {\color{red}\item Kaj so rešitve sistema NDE 1. reda $n$ neznanih funkcij?}
        {\color{red}\item Kaj so rešitve, če je sistem linearen in homogen?}
        {\color{red}\item Kaj so rešitve, če je sistem nehomogen?}
        \item Navedi valovno enačbo
        \item Kaj so njene rešitve?
        {\color{blue}\item Kaj je njena splošna rešitev?}
    \end{enumerate}

    \subsection{Enačba za nihanje strune}
    \begin{enumerate}
        \item Kako opišemo struno?
        \item Kako dobimo obliko strune v fiksnem času?
        \item Kaj predstavlja vrednost funkcije?
        \item Katere sile delujejo na kratkem delčku strune?
        {\color{red}\item Navedi hookov zakon}
        {\color{blue}\item Navedi tangento na krivuljo in njen enotski vektor}
        {\color{blue}\item Kako opišemo dinamiko delčka?}
        {\color{blue}\item Prevedi to enačbo na enačbo valovanja strune}
        {\color{blue}\item Navedi diskreten sistem enačb valovne enačbe }
        {\color{blue}\item Kaj je rešitev?}
        {\color{red}\item Kako iz sistema dobimo valovno enačbo?}
        {\color{red}\item Kaj je torej splošna rešitev valovne enačbe?}
    \end{enumerate}

    \section{Enačbe prvega reda}
    \begin{enumerate}
        \item Definiraj PDE 1. reda
        \item Kako poiščemo rešitve PDE 1. reda v dveh spremenljivkah?
        \begin{itemize}
            {\color{red}\item Kaj iščemo namesto rešitve $u(x, y)$?}
            \item Kaj imamo v vsaki točki te ploskve?
            \item Kako je določena tangentna ravnina?
            {\color{blue}\item Od kod dobimo pogoje za odvoda $u$?}
            \item Kaj je normala tangentne ravnine?
            {\color{red}\item Kaj torej določa naša enačba?}
            {\color{blue}\item Kaj podaja vsaka normala?}
            {\color{blue}\item Kaj torej dobimo?}
        \end{itemize}
    \end{enumerate}

    \subsection{Kvazlinearna PDE 1. reda}
    \begin{enumerate}
        \item Kakšne oblike je kvazlinearna enačba 1. reda?
        {\color{red}\item Razloži kako rešujemo kvazlinearno PDE (v dveh korakih)}
        {\color{red}\item Geometrijsko razloži kako poiščemo rešitev}
    \end{enumerate}

    \section{Metoda karakteristik}
    \begin{enumerate}
        \item Za kaj se uporablja?
        \item Kakšen problem rešujemo?
        \item Kaj je začetni pogoj?
        {\color{red}\item Kakšen pomen ima začetni pogoj?}
        \item Kako se imenuje problem skupaj z začetnim pogojem?
        \item Kaj je splošna rešitev?
        \item Kako lahko prepišemo problem?
        \item Kaj je normala na iskano ploskev?
        \item Kaj nam pove enačba iz vprašanja 7?
        {\color{red}\item Kaj so integralske krivulje polja $(a, b, c)$? Kaj rešijo?}
        {\color{red}\item Kje ležijo integralske krivulje?}
        \item Kako se imenuje sistem?
        \item Koliko rešitev ima karakteristični sistem ob nekem začetnem pogoju?
        {\color{red}\item Kje je definiran začetni pogoj Cauchyjeve naloge? Kaj vzamemo za začetne vrednosti rešitev?}
        \item $x$ in $y$ sta funkciji $t$ in $s$. Zapiši Cauchyjev problem. Kaj so rešitve in kje so definirane.
        \item Skiciraj
        {\color{red}\item Kaj je karakteristična krivulja?}
        {\color{red}\item Kako izmerimo velikost karakteristične krivulje?}
        {\color{red}\item Kaj je rešitev karakterističnega sistema?}
        {\color{red}\item Kdaj je rešitev parametrizacija ploskve?}
        {\color{red}\item Kaj pa če rang ni 2?}
        \item Kaj linearen Cauchyjev problem?
        \item Kako izgleda njegov karakteristični sistem?
        {\color{red}\item Kako rešijemo linearno PDE 1. reda?}
        \color{red}\begin{itemize}
            \item Kaj naredimo najprej?
            \item Kaj naredimo, če s tem dobimo rešitev?
            \item Kaj naredimo, če s tem ne najdemo rešitve?
        \end{itemize}
    \end{enumerate}

    \section{Eksistenčni izrek za kvazlinearne enačbe}
    \begin{enumerate}
        \item Katero nalogo rešujemo?
        \item Navedi tranzverzalnostni pogoj
        {\color{red}\item Kaj je še drugi pogoj, da izrek velja?}
        {\color{red}\item Kaj je bistvo izreka?}
        \item Kaj se spremeni, če transverzalnostni pogoj ni izpolnjen na nekem intervalu za $s$?
    \end{enumerate}

    \subsection{Pomen transverzalnostnega pogoja - 1}
    \begin{enumerate}
        {\color{red}\item Katero preslikavo moramo opazovati?}
        \item Primerjaj transverzalnostni pogoj in odvod preslikave $F$
        \item V kakšni zvezi sta?
        {\color{red}\item Katera vektorja predstavljata vrstici iz transverzalnostnega pogoja?}
        \item Kaj transverzalnostni pogoj pove o teh dveh vektorjih?
        \item Za katera dva vektorja zato velja isto?
        {\color{red}\item Kaj lahko sklepamo?}
    \end{enumerate}

    \subsection{Pomen transverzalnostnega pogoja - 2}
    \begin{enumerate}
        \item Kateri dve determinanti sta enaki? Kolikšna je njuna vrednost?
        \item Po katerem izreku nadaljujemo? Kaj dobimo?
        \item Kaj je karakteristika za rešitev Cauchyjevega problema?
    \end{enumerate}

    \subsection{Dokaz eksistenčnega izreka}
    \begin{enumerate}
        {\color{red}\item Spet navedi Cauchyjev pogoj}
        {\color{red}\item Kaj nam da eksistenčni izrek za NDE 1. reda?}
        \item Katero preslikavo dobimo?
        {\color{red}\item Kje je rang te preslikave maksimalen in koliko je to?}
        \item Zakaj je maksimalen?
        \item Kaj torej ta preslikava parametrizira
        \item Kaj lahko zaradi maksimalnosti ranga definiramo (v obe smeri)?
        \item Izračunaj odvoda $u_x$ in $u_y$
        {\color{red}\item Kaj sledi iz inverznosti preslikav?}
        \item Kateri dve enačbi dobimo
        \item Pokaži, da rešitev obstaja
        {\color{red}\item Kaj želimo dokazati, ko dokazujemo enoličnost rešitve?}
        \item Katero preslikavo definiramo?
        \item Vstavi karakterističi sistem v odvod te preslikave
        \item Kaj dobimo?
        \item Kaj je pri tem začetni pogoj?
        \item Kaj je rešitev NDE?
        \item Kaj je zaključek?
    \end{enumerate}

    \subsection{Neizpolnjevanje tranzverzalnostnega pogoja}
    \begin{enumerate}
        \item Kaj takoj velja?
        \item Kakšna sta lahko stolpca? (2 možnosti)
        {\color{red}\item Kaj velja, če nista vzporedna?}
        {\color{red}\item Koliko je rešitev?}
        {\color{red}\item Kaj velja, če sta vzporedna?}
        {\color{red}\item Koliko je rešitev?}
    \end{enumerate}

    \section{Nelinearna PDE 1. reda}
    \begin{enumerate}
        \item Kakšne oblike je PDE 1. reda?
        {\color{red}\item Kaj nam določa ta enačba v fiksni točki?}
        \item Vpelji nove oznake
        \item Kaj je potem normala?
        {\color{red}\item Kateri pogoj morajo izpolnjevati normale?}
        {\color{red}\item Kakšne ravnine iščemo?}
        {\color{red}\item Kako podamo enačbo take ravnine?}
        {\color{red}\item Kaj nas pri tem zanima?}
        {\color{red}\item Kako je podana ogrinjača te družine?}
        \item Kako se imenuje?
        {\color{red}\item Kako izgleda rešitvena ploskev}
    \end{enumerate}

    \subsection{Parametrizacija Mongeejevega stožca}
    \begin{enumerate}
        \item Odvajaj PDE po $\lambda$
        {\color{red}\item Vpelji 1-parametrično družino vektorskih polj}
        {\color{red}\item Kaj mora za njo veljati?}
        \item Kaj je normala?
        {\color{red}\item Kakšno zvezo med komponentami vektorskega polja dobimo iz druge enačbe?}
        \item Kakšen je sistem za Mongeejev stožec?
        {\color{red}\item Kakšna zveza sledi iz danih sistemov?}
        \item Kaj je parametrizacija Mongeejevega stožca?
        {\color{red}\item Iščemo rešitev $r(t) = (x(t), y(t), u(t))$ enačbe $r' = F$. Zapiši karakteristični sistem}
        \item Pokaži, da iz tega karakterističnega sistema sledi karakterističi sistem za kvazlinearno enačbo
        {\color{red}\item Kako so geometrijsko prikazane normale Mongeejevega stolpca v kvazlinearnem sistemu?}
        {\color{red}\item Kaj nista $p$ in $q$? Kaj sta?}
        \item Kaj moramo narediti, da dobimo dobro definiran sistem?
        {\color{red}\item Geometrično opiši kaj bomo naredili}
        {\color{red}\item Sistem kakšne oblike želimo?}
        \item Zakaj?
        {\color{red}\item Pokaži, da je konstrukcija traku smiselna}
        \item Odvajaj začetno PDE po $x$ in po $y$
        \item Katera odvoda $p$ in $q$ sta enaka? Zakaj? 
        \item Vstavi v odvajano PDE
        {\color{red}\item Čemu sta enaka odvoda $p$ in $q$ po $t$?}
        \item Združi vse v karakteristični sistem 5 enačb
        \item Poimenuj pogoja katerima mora ustrezati začetna krivulja?
        {\color{red}\item Definiraj projekcijo iz $R^5$ na krivuljo v $R^3$}
        {\color{red}\item Navedi kompatibilnostna pogoja}
        {\color{red}\item Navedi tranzverzalnostni pogoj}
    \end{enumerate}

    \subsection{Eksistenčni izrek za PDE 1. reda}
    \begin{enumerate}
        {\color{red}\item Navedi izrek. Kaj so pogoji? Kaj je Bistvo izreka?}
        {\color{red}\item Katera preslikava podaja parametrizacijo ploskve?}
        \item Kaj je rešitev?
        \item Kaj velja za rešitev?
        {\color{red}\item Kako dobimo izražave $u$, $p$ in $q$ v odvisnosti od $x$ in $y$?}
        {\color{red}\item Kakšno rešitev da kompatibilnostni pogoj?}
        {\color{red}\item Kaj zagitavlja izpolnjevanje kompatibilnostnih pogojev?}
        {\color{red}\item Kaj iščemo?}
        {\color{red}\item TRDITEV - Navedi trditev, ki nam to zagotavlja}

        Dokaz
        {\color{red}\item Definiraj preslikavo $F$}
        {\color{red}\item Kaj iščemo? }
        \item Kaj že vemo?
        {\color{red}\item Kateri izrek uporabimo?}
        {\color{red}\item Kaj je njegov pogoj?}
        {\color{red}\item Kaj sledi?}
    \end{enumerate}

    \section{Linearne PDE 2. reda}
    \begin{enumerate}
        \item Navedi 2 razloga zakaj so PDE pomembne
        {\color{red}\item Kakšne oblike je linearna PDE 2. reda? Kaj je na levi in kaj na desni strani?}
        {\color{red}\item Kakšen predpis ima linearni operator?}
        \item Kakšna je linearna PDE 2. reda za dve neznanki?
        \item Definiraj glavni del operatorja $L$
        {\color{red}\item Klasificiraj enačbe 2. reda dveh spremenljivk. Za vsako podaj pogoj}
        {\color{red}\begin{itemize}
            \item Hiperbolična
            \item Parabolična
            \item Eliptična
        \end{itemize}}
        \item Kaj pomeni, da je klasifikacija smiselna?
        {\color{red}\item Kako dobimo vektor novih spremenljivk?}
        {\color{red}\item Kako se izraža rešitev v novih spremenljivkah?}
        \item Izračunaj odvode, ki nastopajo v PDE 2. reda za dve spremenljivki
        \item Kaj je glavni del operatorja $L$ v starih spremenljivkah?
        \item Kaj je glavni del operatorja $L$ v novih spremenljivkah?
        \item Izrazi koeficiente v glavnem delu operatorja v novih spremenljivkah s koeficienti v glavnem delu operatorja v starih spremenljivkah
        {\color{red}\item Kaj predstavljajo te izražave?}
        {\color{red}\item Zapiši kvadratno formo v matrični obliki}
        \item Kaj velja za determinanto?
        \item Kaj zato velja za tip enačbe?
    \end{enumerate}

    \section{Kanonične oblike}
    \begin{enumerate}
        \item Kaj je finta kanonične oblike?
        {\color{red}\item Kaj je kanonična oblika hiperbolične enačbe?}
        \item Kaj je kanonična oblika eliptične enačbe?
        {\color{red}\item Kaj je kanonična oblika parabolične enačbe?}
    \end{enumerate}

    \subsection{Kanonična oblika hiperboličnega tipa}
    \begin{enumerate}
        \item Kaj pravi izrek o kanonični obliki hiperboličnega tipa?
        
        Dokaz
        \item Kako se glasi naša enačba v novih spremenljivkah in kaj želimo?
        {\color{red}\item Kako izrazimo željeno s starimi spremenljivkami?}
        {\color{red}\item Kaj predstavljata dobljeni enačbi?}
        {\color{red}\item Kako rešujemo tako enačbo in kaj dobimo?}
        \item Zapiši enačbo v drugačni obliki
        \item Kaj iščemo in kako to izračunamo?
        \item Kaj sta dobljeni enačbi?
        {\color{red}\item Kako rešujemo dobljeni enačbi?}
        {\color{red}\item Kaj je rešitev?}
        {\color{red}\item Kako se izrazi $\xi$?}
        {\color{red}\item Kaj je geometrični pomen te izražave?}
        \item Kako dobimo drugo kanonično spremenljivko? $\zeta$?
        {\color{red}\item Navedi Triconijevo enačbo}
        \item Kdaj je enačba hiperbolična?
        \item Kaj je njen kanonični sistem?
        \item Poišči njene kanonične koordinate
        \item Zapiši enačbo v kanoničnih koordinatah
    \end{enumerate}

    \subsection{Kanonična oblika paraboličnega tipa}
    \begin{enumerate}
        \item Kaj pravi izrek o kanonični obliki paraboličnega tipa?
        Dokaz
        {\color{red}\item Kaj smemo predpostaviti po definiciji enačbe paraboličnega tipa? Zakaj?}
        \item Kaj moramo pokazati?
        {\color{red}\item Dovolj je pokazati le eno enakost. Zakaj?}
        \item Pretvori enačbo na lažjo obliko
        \item Kako izberemo kanonične koordinate?
        \item PRIMER - Najdi kanonične koordinate enačbe $x^2 * u_{xx} - 2xy * u_{xy} + y^2 * u_{yy} + x * u_x + y * u_y = 0$
    \end{enumerate}

    \subsection{Kanonična oblika eliptičnega tipa}
    \begin{enumerate}
        \item Kaj pravi izrek o kanonični obliki eliptičnega tipa?
        {\color{red}\item Za kakšne koeficiente originalne enačbe iščemo kanonične koordinate?}
        Dokaz
        \item Kakšna je enačba v novih koordinatah in kaj velja za njene koeficiente?
        \item Izrazi nove koeficiente s starimi
        \item Uporabi 3. vprašanje
        {\color{red}\item Definiraj novo funkcijo}
        {\color{red}\item Kateri izraz je ekvavilenten sistemu iz 5. vprašanja}
        {\color{red}\item Za katero funkcijo in izraz velja enako?}
        \item Kako dobimo enačbi prvega reda za ti dve na novo definirani preslikavi?
        \item Kateri enačbi dobimo?
        \item Kako dobimo $\Phi(x, y)$ in $\Psi(x ,y)$?
        {\color{red}\item Kako sta $\Phi(x, y)$ in $\Psi(x ,y)$ povezana z rešitvjo $u$?}
        {\color{red}\item Čemu je ekvavilentna prvotna enačba?}
        {\color{red}\item Kdaj ima definiranje $\Phi(x, y)$ in $\Psi(x ,y)$ smisel?}
        \item Izrazi koorfinati $\xi$ in $\zeta$
        {\color{red}\item Kako iz starih koeficientov PDE dobimo nove? Zapiši v matrini obliki}
        {\color{red}\item Uporabi to v našem primeru in izračunaj kanonično obliko enačbe}
        {\color{red}\item Kaj je translacija pri hiperboličnem tipu?}
        {\color{red}\item Kako jo dobimo?}
        {\color{red}\item Kako se imenujejo translacije?}
    \end{enumerate}

    \section{Podaj primer hiperbolične enačbe}
    \begin{enumerate}
        \item Kaj je njena formula?
        \item Domena
        \item Kanonična oblika?
        \item Kanonični spremenljivki?
        \begin{itemize}
            \item Kaj je karakteristični sistem?
        \end{itemize}
        \item Kaj je kanonična oblika?   
        \item Reši kanonično enačbo
        \item Kaj je originalnih koordinatah?
        {\color{red}\item Definiraj krepko rešitev}
        {\color{red}\item Definiraj šibko rešitev?}
        \item Je limita še vedno 2-krat odvedljiva?
        \item Kaj so karakteristične premice valovne enačbe?
        {\color{red}\item Kakšna informacija se prenaša vzdolž karakteristik neke hiperbolične enačbe?}
        \item Zapiši valovno enačbo na drugačen način
        \item V kakšni zvezi sta dobljena člena?
        {\color{red}\item Kaj velja za rešitev?}
        {\color{red}\item Kaj sta rešitvi "manjših" PDE?}
        {\color{red}\item Kako se imenujeta rešitvi?}
        \item Kako rečemo rešitvi $u = F + G$?
        \item Kako se glasi Cauchyjeva naloga valovne enačbe?
        \item Izrazi $F$ in $G$ z začetnimi pogoji
        \item Kaj je torej rešitev?
        \item Kako se imenuje?
        {\color{red}\item IZREK - Podaj eksistenčni izrek za valovno enačbo}
        {\color{red}\item Kaj pomeni dobra pogojenost?}
        Dokaz
        \item Kako dokažemo obstoj in enoličnost?
        {\color{red}\item Kdaj je rešitev dvakrat zvezno odvedljiva?}
        {\color{red}\item Kaj pomeni zvezna odvisnost?}
        \item Pokaži, da je res zvezno odvisna od začetnih pogojev
    \end{enumerate}

    \section{Nehomogena valovna enačba}
    \begin{enumerate}
        \item Cauchyjev problem nehomogene valovne enačbe
        {\color{red}\item Kaj v homogeni enačbi vpliva na vrednosti njene rešitve?}
        \item Kje je definirana rešitev homogene?
        \item Kaj je karakteristični trikotnik?
        \item Kje je definirana rešitev nehomogene?
        {\color{red}\item TRDITEV - Koliko rešitev ima nehomogen problem?}
        {\color{red}\item Dokaz}
        {\color{red}\begin{itemize}
            \item Definiraj novo funkcijo in zapiši katero enačbo rešitev
            \item Kaj so začetni pogoji te funkcije?
            \item Po čem sledi enoličnost rešitve nehomogene?
        \end{itemize}}
        {\color{red}\item IZREK - Greenov izrek (formula)}
        \item Izračunaj drugi integral nehomogenega dela po karakterističnem trikotniku
        \begin{itemize}
            {\color{red}\item Uporabi greenovo formulo}
            {\color{red}\item Izračun po bazi}
            {\color{red}\item Izračun po desni strani}
            {\color{red}\item Izračun po levi strani}
            \item Združi v eno vrstico
            \item Kaj je dobljena rešitev?
        \end{itemize}
    \end{enumerate}

    \subsection{Duhamelov postopek}
    \begin{enumerate}
        \item Kaj je finta postopka?
        {\color{red}\item TRDITEV - S pomočjo rešitve katerega problema znamo rešiti našo valovno enačbo pri pogojih $u = 0$ in $u_t = 0$?}
        Dokaz
        \item Označi integral pomožne funkcije
        \item Kaj so začetni pogoji te funkcije?
        {\color{red}\item Katero PDE porodi?}
        {\color{red}\item Kaj je rešitev za $v$?}
        {\color{red}\item Kaj je rešitev zaćetnega problema za $u_{crta}$?}
        \item Kateri splošnejši problem želimo po navadi rešiti?
        \item Kaj je njegova rešitev?
        \item Zapiši formulo rešitve
    \end{enumerate}

    \section{Uporaba Fourierove analize pri reševanju PDE}
    \begin{enumerate}
        {\color{red}\item Kaj je nastavek za rešitev?}
        {\color{red}\item Kako lahko opišemo našo rešitev?}
        \item Navedi toplotno enačbo na končnem nosilcu
        {\color{red}\item Kje je definirana rešitev?}
        {\color{red}\item Navedi Dirichletov robni pogoj}
        \item Kaj je začetni pogoj
        \item Ob skici pojasni kaj imamo podano in kaj želimo zračunati
        \item S kakšnim nastavkom rešujemo enačbo?
        \item Prevedi toplotno enačbo na sistem enačb
        {\color{red}\item Kako se imenuje sistem?}
        \item Kaj je prostorski del našega problema? Kaj je robni pogoj?
        \item Katere tri možnosti ločimo?
        \item Kaj je splošna rešitev prostorskega dela pri $\lambda < 0$?
        \item Kaj je rešitev?
        \item Kaj je splošna rešitev prostorskega dela pri $\lambda = 0$?
        \item Kaj je rešitev?
        \item Kaj je splošna rešitev prostorskega dela pri $\lambda > 0$?
        \item Kaj je rešitev?
        \item Kaj smo reševali?
        \item Kaj so lastne vrednosti?
        \item Kaj so lastni vektorji?
        {\color{red}\item Kaj so rešitve časovnega dela?}
        {\color{red}\item Kaj je rešitev in zakaj?}
        \item Katero rešitev moramo poiskati?
        \item Kako razvijemo začetni problem v vrsto?
        {\color{red}\item Kaj so koeficienti v razvoju začetnega problema?}
        \item Kaj je torej rešitev?
        {\color{red}\item Kaj pomeni, da je rešitev krepka?}
        \item S čem lahko dokažemo krepkost?
        \item Kaj je Neumannov pogoj za začetno enačbo?
    \end{enumerate}

    \subsection{Neomogeni začetno-robni problemi}
    \begin{enumerate}
        \item Navedi sistem za nehomogen začetno robni problem valovne enačbe
        \item Ali lahko rešujemo s seperacijo spremenljivk?
        \item Kako lahko opazujemo homogen sistem?
        {\color{red}\begin{itemize}
            \item Kot kakšno funkcijo smo opazovali rešitev problema?
            \item Kako lahko zapišemo krivulje v $R^3$?
            \item Kako zapišemo krivuljo v $R^3$ glede na neko bazo?
            \item Kaj so baze v prostoru spremenljivke $x$?
            \item Kako lahko zapišemo rešitev glede na bazo?
        \end{itemize}}
        {\color{red}\item S kakšnim nastavkom lahko rešujemo nehomogeno enačbo?}
        \item Kako?
        {\color{red}\item Čemu ustreza desna stran enačbe?}
        \item Najdi koeficiente v razvoju rešitve odvisne od $t$
        {\color{red}\item Kako določimo neznane koeficiente?}
        {\color{red}\item Enačbe kakšne oblike smo reševali?}
        \item Kaj je $\mathcal{L}$
        {\color{red}\item Kaj je definicijsko območje, zaloga prednosti in predpis za $\mathcal{L}$}
        {\color{red}\item Kaj smo dobili?}
        \item Kaj pomeni KONS?
        {\color{red}\item Kaj predpostavljamo za lastne vektorje?}
        {\color{red}\item Kako lahko zato izrazimo funkcijo definirano na prostosu od $x$?}
        \item Kako lahko predstavimo rešitev?
        \item Vstavi rešitev v toplotno reačbo
        \item Kaj dobimo?
        {\color{red}\item Kdaj znamo rešiti naš problem?}
        {\color{red}\item Kaj je Sturm-Lionvilleeva teorija?}
    \end{enumerate}

    \subsection{Toplotna enačba na neskončnem nosilcu}
    \begin{enumerate}
        \item Katero metodo uporabljamo pri tem?
        {\color{red}\item Kako lahko razvijemp periodično funkcijo v vrsto?}
        {\color{red}\item Kaj je baza? Na katerem prostoru?}
        {\color{red}\item Kaj so koeficienti v vrsti?}
        {\color{red}\item Vpelji novo oznako funkcije}
        {\color{red}\item Vpelji oznako za spremenljivko}
        \item Kaj je razlika med dvema zaporednima spremenljivkama?
        \item Čemu je enaka začetna funkcija?
        {\color{red}\item Kako se imenuje vsota, ki jo dobimo?}
        {\color{red}\item Zapiši $f$ z integralom}
        {\color{red}\item Definiraj Fourierovo transformacijo}
        {\color{red}\item Definiraj inverzno Fourierovo transformacijo}
        {\color{red}\item Definiraj Schwarzov razred}
        {\color{red}\item Kaj velja za Fourierovo transformacijo? (2)}
        \item Kakšen začetni problem želimo rešiti s Fourierovo transformacijo?
        {\color{red}\item Kako dobimo integralsko jedro, če je začetni problem definiran na Schwarzovem razredu?}
    \end{enumerate}

    \section{Diracova delta "funkcija"}
    \begin{enumerate}
        {\color{red}\item Kako generiramo Diracovo delta funkcijo? }
        {\color{red}\begin{itemize}
            \item Kaj je nosilec?
            \item integral
            \item (pol)
        \end{itemize}}
        \item Opis $\delta_0 (x)$
        \item Kako bi to intuitivno opisal? 
        \item Nariši vse
        {\color{red}\item Kako pridemo do evaluacije?}
        {\color{red}\begin{itemize}
            \item integral
            \item Kaj velja za  $\delta_0 (x)$?
            \item Zakaj to velja?
            \item Oznaka evaluacije
        \end{itemize}}
        {\color{red}\item Kaj v resnici definicija evaluacije?}
        {\color{red}\item Kako imenujemo  $\delta_0 (x)$?}
        {\color{red}\item Kako bi laho zapisali evaluacijo kot skalarni produkt?}
        \item Po kateri mertiki?
        {\color{red}\item Podaj primer}
        {\color{red}\begin{itemize}
            \item Kaj je njena *posebnost*
        \end{itemize}}
        {\color{red}\item Definiraj prostor testnih funkcij}
        \item Definiraj prostor distirbucij
        \item Prostor česa je torej to? Kaj je to?
        {\color{red}\item Podaj primer elementa iz prostora }
        {\color{red}\item Podaj primer - REGULARNA DISTRIBUCIJA}
        {\color{red}\item Definiraj odvod distribucije }
        \item Kaj je odvod regularne distribucije? 
        {\color{red}\item Definiraj odvod  $\delta_0 (x)$}
        {\color{red}\item Definiraj  $\delta_0 (x)$ s polom v poljubni točki}
    \end{enumerate}

    \section{Integralska jedra}
    \begin{enumerate}
        {\color{red}\item Kje do zdaj se skriva linearen operator?}
        {\color{red}\item Definiraj jedro}
        {\color{red}\item Kakšen linearen operator vzamemo?}
        \item Po kakšnem integralu integriramo?
        \item Kaj je disktretna verzija integralskega jedra?
        \item Zapiši ekvavilentna matrična zapisa (integralskega in matričnega)
        \item TRDITEV - Je Diracova funkcija soda ali liha? 
        {\color{red}\item Dokaži}
        \item {\color{red}Kaj sledi iz tega?} Pokaži, da res
    \end{enumerate}

    \subsection{Iskanje toplotnega jedra}
    \begin{enumerate}
        \item Kaj je toplotna enačba?
        {\color{red}\item S kakšnim nastavkom jo rešujemo?}
        \item Vstavi v enačbo in posploši
        \item Izrazi koeficiente nastavka
        \item Kako dobimo konstanto, ki nastopa v koeficientu iz nastavka?
        {\color{red}\item Izračunaj jo}
        \item Kaj je torej rešitev?
        \item Kaj je najenostavnejši začetni pogoj za to rešitev?
        \item Izračunaj rešitev
        {\color{red}\item Polepšaj zapis rešitve}
        {\color{red}\item Kaj predstavlja rešitev?}
        {\color{red}\item TRDITEV - Kaj je rešitev $u(x - a, t)$? Dokaz}
        \item Kaj je rešitev začetnega problema, če je začetni pogoj poljubna delta funkcija?
        {\color{red}\item TRDITEV - Kaj je posledica linearnosti toplotne enačbe? Dokaz} 
        \item Kako lahko zapišemo rešitev splošnega začetnega problema 
        {\color{red}\item Kako lahko drugače zapišemo $f(x)?$}
        {\color{red}\item Kaj je torej rešitev?}
        {\color{red}\item Kako bi opisala kaj smo dobili?}
        {\color{red}\item Kaj v tem je toplotno jedro?}
    \end{enumerate}

    \section{Eliptična enačba}
    \begin{enumerate}
        {\color{red}\item S kakšnimi enačbami se bomo ukvarjali? (2)}
        \item Poimenuj ju
        \item Kako je definiran Laplacov operator?
        \item Kaj pa v več dimenzijah?
        \item Kaj opisujeta enačbi?
        {\color{red}\item Kaj je enačba nehomogene toplotne enačbe v dveh spremenljivkah?}
        {\color{red}\item Kje je podan robni pogoj?}
        {\color{red}\item Kaj pomeni, da je problem stacionaren (pri velikih časih)?}
        {\color{red}\item Kako opišemo stacionarno stanje?}
        {\color{red}\item Kaj je enačba eliptičnega začetnega problema? }
        \item Naštej tri vrste robnih problemov
        {\color{red}\item Opiši Dirichletov problem}
        \item Opiši Neumannov problem
        \item Kaj je Neumannov odvod?
        {\color{red}\item Kakšna je normala?}
        \item Ali so lahko podatki Neumannovega problema povsem poljubni?
        {\color{red}\item Opiši mešani problem}
        {\color{red}\item Kdaj je Neumannov problem rešljiv?}
        {\color{red}\item Iz katerega zreka to sledi?}
        {\color{red}\item Navedi ta izrek}
        {\color{red}\item V kaj pretvorimo izrek v našem primeru?}
        {\color{red}\item Kaj mora torej veljati v Neumannovem primeru? Poimenuj}
        {\color{red}\item Kaj mora veljati v Dirichletovem primeru?}
        \item Definiraj harmonično funkcijo
        {\color{red}\item Kaj velja za vse harnomične funkcije?}
        \item Definiraj rotacijsko simetričnost. Kaj to intuitivno pomeni?
        {\color{red}\item Zapiši laplacov operatov v polarnih koordinatah}
        \item Kako se torej glasi laplacova enačba?
        \item Kaj pa če je $W$ rotacijsko simetrična?
        {\color{red}\item Kaj je rešitev?}
        {\color{red}\item Kako se rešitev imenuje?}
    \end{enumerate}

    \subsection{Princip maksima}
    \begin{enumerate}
        \item IZREK - krepki princip maksima
        \item IZREK - šibki princip maksima
        {\color{red}\item IZREK - princip sfecičnih povprečij}
        {\color{red}\item Dokaz principa sferičnih povprečij}
        \begin{itemize}
        {\color{red}    \item Vpelji funkcijo}
        {\color{red}    \item Kaj hočemo pokazat?}
        {\color{red}    \item Kaj bo iz tega sledilo?}
            \item Dokaz 1
            {\color{red}\item Dokaz 2}
        \end{itemize}
        \item Kaj je obrat izreka sferičnih povprečij?
        \item Dokaz obrata
        \begin{itemize}
            \item Predpostavi, da ni harmonična. Kje?
            {\color{red}\item Na operatorju (integral) uporabi Greenovo formulo}
            {\color{red}\item S čem je dobljeno v protislovju?}
        \end{itemize}
        \item Navedi krepki princip maksima
        \item Dokaz
        \begin{itemize}
            \item Nariši skico
            {\color{red}\item Kaj hočemo dokazat?}
            \item Predpostavi nasprotno - dobimo dve možnosti
            \item Katera možnost takoj odpove? Zakaj?
            {\color{red}\item Zakaj druga možnost ne velja?}
            \item Kako dobimo, da je u konstantna na celotnem območju?
        \end{itemize}
    \end{enumerate}

    \subsection{Primeri uporabe principa maksima}
    \begin{enumerate}
        {\color{red}\item IZREK - Kdaj ima Dirichletov problem največ eno rešitev?}
        \item Dokaz
        \begin{itemize}
            {\color{red}\item Def novo funkcijo}
            \item Je harmonična? Zakaj?
            {\color{red}\item Koliko sta njen min in max?}
        \end{itemize}
        {\color{red}\item Kaj pa če območje ni omejeno? Podaj primer z več kot eno rešitvijo}
        \item IZREK - od česa je odvisna rešitev?
        \item Dokaz
        \begin{itemize}
            {\color{red}\item Definiraj dva problema in novo funkcijo z njima}
            \item Kaj je problem za novo funkcijo?
            {\color{red}\item Kaj naj velja za robna pogoja?}
            {\color{red}\item Kaj potem sledi za novo funkcijo?}
        \end{itemize}
    \end{enumerate}

    \section{Osnove Greenove teorije}
    \begin{enumerate}
        \item Pri čem se uporablja?
        {\color{red}\item Katere probleme rešuje?}
        {\color{red}\item Kako jih rešuje?}
        {\color{red}\item Kaj iščemo v Dirichletovem primeru? Kje sta definirani?}
        {\color{red}\item Kaj mora zanju veljati?}
    \end{enumerate}

    \subsection{Greenove identitete}
    \begin{enumerate}
        {\color{red}\item Iz česa sledijo greenove identitete?}
        {\color{red}\item Kaj je 1. Greenova identiteta?}
        {\color{red}\item Kaj je 2. Greenova identiteta?}
        \item Kaj je 3. Greenova identiteta?
        {\color{red}\item IZREK - Posledica česa je?}
        \begin{itemize}
            {\color{red}\item Kaj so pogoji?}
            \item Koliko rešitev ima Dirichletov problem?
            \item Koliko rešitev ima mešani problem?
            {\color{red}\item Koliko rešitev ima Neumannov problem in kakšne oblike so?}
        \end{itemize}
        \item Dokaz za Dirichletov
        \item Dokaz za mešani
        \begin{itemize}
            \item Definiraj novo funkcijo
            \item Kaj je njen začetni problem?
            {\color{red}\item Kaj dobimo z 2. Greenovo identiteto?}
            {\color{red}\item Kaj sledi za normalo gradienta nove funkcije?}
            {\color{red}\item Kaj sledi za to novo funkcijo?}
        \end{itemize}
        \item Dokaz za Neumannov problem
        \begin{itemize}
            \item Kaj hočemo dokazat?
            \item Definiraj novo funkcijo
            \item Kaj je njen začetni problem?
            \item Vstavi v 2. Greenovo identiteto
            \item Kaj velja za gradient nove funkcije?
            \item Kakšna je torej nova funkcija?
        \end{itemize}
    \end{enumerate}

    \subsection{Greenova funkcija}
    \begin{enumerate}
        \item Kaj je motivacija?
        DIRICHLETOV PROBLEM
        {\color{red}\item Kaj je rešitev fundamentalna Laplacove enačbe?}
        {\color{red}\item Kaj je rešitev Laplacove enačbe, ki je definirana povsod razen v eni točki}
        {\color{red}\item Kakšna je ta rešitev?}
        {\color{red}\item Kaj pa velja za $u$?}
        \item Vpelji oznako rešitve
        {\color{red}\item Kako bomo iskali jedro?}
        {\color{red}\item Kakšna mora biti $u$ in kakšen $\Omega$?}
        {\color{red}\item Kje lahko zapišemo 3. Greenovo formujo za $u$? Formalno zapiši to območje}
        \item Uporabi 3. Greenovo identiteto za $u$ in $\Gamma$ na tem območju
        {\color{red}\item Uporabi vse kar veš o območjih in funkcijah, ki nastopajo v tej enačbi}
        KAJ DOBIMO KO GRE $\epsilon \Rightarrow 0$?
        \item Kaj je fundamentalna rešitev v polarnih koordinatah?
        \item Kaj je rešitev za točke iz roba kroga, ki smo ga odvzeli?
        {\color{red}\item Kako se glasi prvi člen - integral po robu kroga}
        \item Kaj dobimo v limiti?
        {\color{red}\item Kako se glasi drugi člen - integral po robu kroga?}
        \item Kaj dobimo v limiti?
        \item Kaj je torej rešitev?
        {\color{red}\item Kako se imenuje rešitev?}
        {\color{red}\item Kakšna bo torej rešitev za naš (Dirichletov) problem?}
        \item Kako bomo modificirali funkcije, da bomo rešitev izrazili z že znanimi funkcijami?
        \item Koliko funkcij rabimo?
        {\color{red}\item Definiraj jih}
        \item Definiraj preslikavo, ki jo bomo uporabili v rešitvi
        {\color{red}\item Zapiši Greenovo reprezentijsko formulo za $u$, $h$ in $\Gamma$}
        {\color{red}\item Izrazi rešitev $u$ z $G_{\Omega}$}
        {\color{red}\item IZREK - Povzetek}
        \item Navedi Greenovo reprezentijsko formulo s funkijo $G_{\Omega}$ 
        \item Kako dokažemo?
    \end{enumerate}
    
    \subsection{Fundamentalne rešitve}
    \begin{enumerate}
        {\color{red}\item Kako se glasi Greenova reprezentijska formula za testno funkcijo?}
        {\color{red}\item Navedi nekoliko splošnejšo obliko}
        \item Razpiši
        {\color{red}\item Kako integriramo naprej?}
        {\color{red}\item Kaj potem dobimo?}
        {\color{red}\item Kako se dobljen rezultat vede?}
        {\color{red}\item Čemu je torej ta integral enak?}
        {\color{red}\item Kaj pa velja za $grad(\Gamma)$?}
        {\color{red}\item Definiraj fundamentalno rešitev Laplacove enačbe}
        {\color{red}\item Navedi še boljšo definicijo fundamentalne rešitve}
        {\color{red}\item Koliko je fundamentalnih rešitev? Dokaži}
        {\color{red}\item Podaj primer fundamentalne rešitve}
        {\color{red}\item Definiraj adjungirana operatorja}
        \item Podaj primer sebi adjungiranega operatorja (2x) 
        \item Podaj primer operatorja, ki ni samemu sebi adjungiran
        {\color{red}\item Definiraj fundamentalno rešitev z adjungiranim operatorjem}
        {\color{red}\item Čemu je enak parcialni diferencialni operator uporabljen na fundamentalni rešitvi?}
    \end{enumerate}

    \section{Neumannova funkcija}
    \begin{enumerate}
        \item Kaj je motivacija?
        \item Pokaži, da Neumanove funkcije ne moremo dobiti na isti način kot Greenove
        \begin{itemize}
            \item Kaj je Greenova reprezentijska formula?
            \item S čem smo nadomestili $\Gamma$?
            {\color{red}\item Uporabi kompatibilnostni pogoj}
            {\color{red}\item Čemu je torej enaka rešitev?}
            {\color{red}\item Kaj dobimo, če je $u = 1$?}
        \end{itemize}
        {\color{red}\item Kako popravimo robni pogoj za $h$?}
        \item Definiraj Neumannovo funkcijo
        {\color{red}\item Kaj je rešitev Neumannovega problema?}
        \item Zakaj se rešitve razlikujejo le za konstanto?
        \item Poišči dve rešitvi Neumannovega problema
    \end{enumerate}

    \section{Greenova funkcija in konformne preslikave}
    \begin{enumerate}
        \item Kaj je motivacija?
        {\color{red}\item Definiraj konformno preslikavo}
        {\color{red}\item Definiraj kompleksno odvedljivost}
        \item Kako razdelimo kompleksno odvedljivo funkcjo?
        {\color{red}\item Definiraj odvod te razdelitve}
        \item Kaj je matrika odvoda?
        {\color{red}\item Zapiši "realno" definicijo odvoda}
        {\color{red}\item Zapiši jo v realni obliki}
        {\color{red}\item Zapiši sistem iz odvodov $\phi$}
        {\color{red}\item Kakšen sistem sestavljata $\phi_R$ in $\phi_C$?}
        {\color{red}\item Kaj sledi iz tega za realni in imaginarni del funkcije?}
        {\color{red}\item IZREK - Med kakšnimi območji obstajajo konformne preslikave?}
        \item V katero območje lahko vedno slikamo? aka podaj primer takega območja
        {\color{red}\item IZREK - Definiraj preslikavo in zapiši Greenovo funkcijo za Dirichletov problem}
        Dokaz
        \item Kaj velja zaradi konformnosti preslikave iz izreka?
        {\color{red}\item Kako jo lahko zapišemo drugače?}
        {\color{red}\item Kaj velja za novo funkcijo v zapisu? Pokaži}
        {\color{red}\item Ali je $\Psi(\xi, \xi) = 0$? Zakaj?}
        {\color{red}\item Razpiši desno stran greenove funkcije}
        \item Kako bi torej zapisali Greenovo funkcijo?
        \item Kaj velja za $h$ v tem zapisu? 
        {\color{red}\item Kaj še moramo dokazati? (2)}
        \item 1
        \begin{itemize}
            \item Kaj sledi iz konformnosi?
            \item Kam se slika rob?
            \item Kakšna je torej preslikava med roboma?
            {\color{red}\item Koliko je Greenova funkcija enaka?}
        \end{itemize}
        \item 2
        \begin{itemize}
            {\color{red}\item Kaj je $h$?}
            {\color{red}\item Kateri del $h$ oz. $\Psi$ mora biti harmoničen?}
            {\color{red}\item Razpiši $\Psi$}
            {\color{red}\item Zakaj je $\Psi$ holomorfna?}
        \end{itemize}
        {\color{red}\item Kako iz konformne preslikave dobiš preslikavo iz izreka?}
    \end{enumerate}

    \section{Valovna enačba v $R^3$}
    \begin{enumerate}
        \item Kaj je valovna enačba v 1+1?
        \item Kaj je rešitev valovne enačbe v 1+1?
        \item Kaj je rešitev, če je $f = 0$?
        \item Nariši skico vala 
        \item 
        {\color{red}\item Kako imenujemo valovno enačbo v $R^3$} 
        \item Kaj je valovna enačba v $R^3$?
        \item Kakšna morata biti začetna pogoja?
        \item Zapiši problem za valovno enačbo v sferičnih koordinatah
        \item Kakšne rešitve iščemo?
        \item Na kaj se pretvori valovna enačba?
        {\color{red}\item Kako jo pretvorimo na standardno valovno enačbo?}
        \item Kaj želimo rešiti s pomočjo tega?
        \item Kaj je problem pri reševanju?
        \item Kako razširimo začetna pogoja?
        {\color{red}\item Kaj je torej rešitev?}
        \item Kaj je splošen problem?
        {\color{red}\item LEMA - Kako sta povezani rešitvi problemov, ki imata ničeln $f$ oz. $g$?}
        {\color{red}\item Dokaz}
        \item Kako razdelimo problem valovne enačbe na dva podproblema?
        {\color{red}\item Definiraj sferično povprečje}
        {\color{red}\item DARBOUXOV IZREK}
        Dokaz
        \item Zakaj hočemo sferično povprečje zapisati v lepši obliki?
        {\color{red}\item Zapiši sferično povprečje v lepši obliki}
        \item Koliko je odvod sferičnega povprečja po $a$?
        \item Kako vidimo, da res reši enačbo?
        {\color{red}\item Kateri začetni problem reši sferično povprečje?}
        \item Kakšen operator je sferično povprečje?
        \item Koliko je sferično povprečje pri radiju 0?
        \item IZREK - Če u repi valovno enačbo v 3D, katero enačbo reši sferično povprečje za $u$?
        {\color{red}\item Dokaz}
        \item Če $u$ reši valovno enačbo za $f = 0$, Katero enačbo reši $M_u$ in pri katerem začetnem pogoju?
        \item Kaj je tu parameter in kaj aktivna spremenljivka?
        \item Katero enačbo in pri katerem začetnem pogoju reši $M_u$ po izreku?
        \item Kaj je tu parameter in kaj aktivna spremenljivka?
        \item Kako razširimo na celo realno os?
        {\color{red}\item Kako se glasi rešitev?}
        {\color{red}\item Kako pa dobimo rešitev $u$?}
        {\color{red}\item Kako se glasi rešitev $u$?}
        \item ?
    \end{enumerate}

\end{document}