\documentclass{article}
\usepackage{amsmath}
\usepackage{xcolor}
\title{Diferencialne enačbe}

\begin{document}
    \maketitle
    
    \section{Uvod}
    \begin{enumerate}
        \item Definiraj parcialno diferencialno enačbo
        \item Definiraj red enačbe
        \item Kdaj imamo eno funkcijo in kdaj sistem funkcij?
        \item Kaj je rešitev PDE?
        \item Navedi tri oblike PDE in opiši vsako on njih
        \begin{itemize}
            \item Linearna
            \item Semilinearna
            \item Kvazlinearna
        \end{itemize}
        \item Navedi izreke o rešljivosti PDE
        \item Koliko rešitev je v splošnem? Kako dobimo eno samo rešitev?
        \item Definirajs $F$ kot vektorsko polje. Kaj velja za rešitev PDE? (zakaj?)
        \item Opiši rešitev PDE geometrijsko
        \begin{itemize}
            \item Kak graf pripada funkciji u? Kaj je to geometrijsko?
            \item Kaj je definirano na tej ploskvi?
            \item Kaj sta tangenti na krivulji v neki točki?
            \item Čemu je enaka linearna ogrinjača teh dveh tangent?
        \end{itemize}
        \item Definiraj - povzami - kaj je tangentna ravnina?
        \item Kaj podaja vsak par funkcij $f$ in $g$?
        \item Kako je polje invariantno?
        \item Kakšen je predpis za polje ravnin?
        \item Definiraj integrabilno polje
        \item Kaj je avtonomen sistem?
        \item Kaj je avtonomen sistem NDE 1. reda?
        \item Kako ga še lahko zapišemo?
        \item Kaj je rešitev sistema?
        \item Kaj torej velja za tangente funkcije gama?
        \item Kako imenujemo funkcijo gama?
        \item Kaj pove eksistenčni izrek o tej krivulji?
        \item Velja naj podobnega za polje ravnin? 
        \item Navedi Stokesov izrek na ektorskem polju $(f(x, y), g(x, y))$
        \item Kdaj je polje potencialno?
        \item Kaj velja na majhnih območjih?
    \end{enumerate}

    \section{Število rešitev PDE}
    \begin{enumerate}
        \item Kaj so rešitve sistema NDE 1. reda $n$ neznanih funkcij?
        \item Kaj so rešitve, če je sistem linearen in homogen?
        \item Kaj so rešitve, če je sistem nehomogen?
        \item Navedi valovno enačbo
        \item Kaj so njene rešitve?
        \item Kaj je njena splošna rešitev?
    \end{enumerate}

    \subsection{Enačba za nihanje strune}
    \begin{enumerate}
        \item Kako opišemo struno?
        \item Kako dobimo obliko strune v fiksnem času?
        \item Kaj predstavlja vrednost funkcije?
        \item Katere sile delujejo na kratkem delčku strune?
        \item Navedi hookov zakon
        \item Navedi tangento na krivuljo in njen enotski vektor
        \item Kako opišemo dinamiko delčka?
        \item Prevedi to enačbo na enačbo valovanja strune
        \item Navedi diskreten sistem enačb valovne enačbe 
        \item Kaj je rešitev?
        \item Kako iz sistema dobimo valovno enačbo?
        \item Kaj je torej splošna rešitev valovne enačbe?
    \end{enumerate}

    \section{Enačbe prvega reda}
    \begin{enumerate}
        \item Definiraj PDE 1. reda
        \item Kako poiščemo rešitve PDE 1. reda v dveh spremenljivkah?
        \begin{itemize}
            \item Kaj iščemo namesto rešitve $u(x, y)$?
            \item Kaj imamo v vsaki točki te ploskve?
            \item Kako je določena tangentna ravnina?
            \item Od kod dobimo pogoje za odvoda $u$?
            \item Kaj je normala tangentne ravnine?
            \item Kaj torej določa naša enačba?
            \item Kaj podaja vsaka normala?
            \item Kaj torej dobimo?
        \end{itemize}
    \end{enumerate}

    \subsection{Kvazlinearna PDE 1. reda}
    \begin{enumerate}
        \item Kakšne oblike je kvazlinearna enačba 1. reda?
        \item Razloži kako rešujemo kvazlinearno PDE (v dveh korakih)
        \item Geometrijsko razloži kako poiščemo rešitev
    \end{enumerate}

    \section{Metoda karakteristik}
    \begin{enumerate}
        \item Za kaj se uporablja?
        \item Kakšen problem rešujemo?
        \item Kaj je začetni pogoj?
        \item Kakšen pomen ima začetni pogoj?
        \item Kako se imenuje problem skupaj z začetnim pogojem?
        \item Kaj je splošna rešitev?
        \item Kako lahko prepišemo problem?
        \item Kaj je normala na iskano ploskev?
        \item Kaj nam pove enačba iz vprašanja 7?
        \item Kaj so integralske krivulje polja $(a, b, c)$? Kaj rešijo?
        \item Kje ležijo integralske krivulje?
        \item Kako se imenuje sistem?
        \item Koliko rešitev ima karakteristični sistem ob nekem začetnem pogoju?
        \item Kje je definiran začetni pogoj Cauchyjeve naloge? Kaj vzamemo za začetne vrednosti rešitev?
        \item $x$ in $y$ sta funkciji $t$ in $s$. Zapiši Cauchyjev problem. Kaj so rešitve in kje so definirane.
        \item Skiciraj
        \item Kaj je karakteristična krivulja?
        \item Kako izmerimo velikost karakteristične krivulje?
        \item Kaj je rešitev karakterističnega sistema?
        \item Kdaj je rešitev parametrizacija ploskve?
        \item Kaj pa če rang ni 2?
        \item Kaj linearen Cauchyjev problem?
        \item Kako izgleda njegov karakteristični sistem?
        \item Kako rešijemo linearno PDE 1. reda?
        \begin{itemize}
            \item Kaj naredimo najprej?
            \item Kaj naredimo, če s tem dobimo rešitev?
            \item Kaj naredimo, če s tem ne najdemo rešitve?
        \end{itemize}
    \end{enumerate}

    \section{Eksistenčni izrek za kvazlinearne enačbe}
    \begin{enumerate}
        \item Katero nalogo rešujemo?
        \item Navedi tranzverzalnostni pogoj
        \item Kaj je še drugi pogoj, da izrek velja?
        \item Kaj je bistvo izreka?
        \item Kaj se spremeni, če transverzalnostni pogoj ni izpolnjen na nekem intervalu za $s$?
    \end{enumerate}

    \subsection{Pomen transverzalnostnega pogoja - 1}
    \begin{enumerate}
        \item Katero preslikavo moramo opazovati?
        \item Primerjaj transverzalnostni pogoj in odvod preslikave $F$
        \item V kakšni zvezi sta?
        \item Katera vektorja predstavljata vrstici iz transverzalnostnega pogoja?
        \item Kaj transverzalnostni pogoj pove o teh dveh vektorjih?
        \item Za katera dva vektorja zato velja isto?
        \item Kaj lahko sklepamo?
    \end{enumerate}

    \subsection{Pomen transverzalnostnega pogoja - 2}
    \begin{enumerate}
        \item Kateri dve determinanti sta enaki? Kolikšna je njuna vrednost?
        \item Po katerem izreku nadaljujemo? Kaj dobimo?
        \item Kaj je karakteristika za rešitev Cauchyjevega problema?
    \end{enumerate}

    \subsection{Dokaz eksistenčnega izreka}
    \begin{enumerate}
        \item Spet navedi Cauchyjev pogoj
        \item Kaj nam da eksistenčni izrek za NDE 1. reda?
        \item Katero preslikavo dobimo?
        \item Kje je rang te preslikave maksimalen in koliko je to?
        \item Zakaj je maksimalen?
        \item Kaj torej ta preslikava parametrizira
        \item Kaj lahko zaradi maksimalnosti ranga definiramo (v obe smeri)?
        \item Izračunaj odvoda $u_x$ in $u_y$
        \item Kaj sledi iz inverznosti preslikav?
        \item Kateri dve enačbi dobimo
        \item Pokaži, da rešitev obstaja
        \item Kaj želimo dokazati, ko dokazujemo enoličnost rešitve?
        \item Katero preslikavo definiramo?
        \item Vstavi karakterističi sistem v odvod te preslikave
        \item Kaj dobimo?
        \item Kaj je pri tem začetni pogoj?
        \item Kaj je rešitev NDE?
        \item Kaj je zaključek?
    \end{enumerate}

    \subsection{Neizpolnjevanje tranzverzalnostnega pogoja}
    \begin{enumerate}
        \item Kaj takoj velja?
        \item Kakšna sta lahko stolpca? (2 možnosti)
        \item Kaj velja, če nista vzporedna?
        \item Koliko je rešitev?
        \item Kaj velja, če sta vzporedna?
        \item Koliko je rešitev?
    \end{enumerate}

    \section{Nelinearna PDE 1. reda}
    \begin{enumerate}
        \item Kakšne oblike je PDE 1. reda?
        \item Kaj nam določa ta enačba v fiksni točki?
        \item Vpelji nove oznake
        \item Kaj je poten normala?
        \item Kateri pogoj morajo izpolnjevati normale?
        \item Kakšne ravnine iščemo?
        \item Kako podamo enačbo take ravnine?
        \item Kaj nas pri tem zanima?
        \item Kako je podana ogrinjača te družine?
        \item Kako se imenuje?
        \item Kako izgleda rešitvena ploskev
    \end{enumerate}

    \subsection{Parametrizacija Mongeejevega stožca}
    \begin{enumerate}
        \item Odvajaj PDE po $\lambda$
        \item Vpelji 1-parametrično družino vektorskih polj
        \item Kaj mora za njo veljati?
        \item Kaj je normala?
        \item Kakšno zvezo med komponentami vektorskega polja dobimo iz druge enačbe?
        \item Kakšen je sistem za Mongeejev stožec?
        \item Kakšna zveza sledi iz danih sistemov?
        \item Kaj je parametrizacija Mongeejevega stožca?
        \item Iščemo rešitev $r(t) = (x(t), y(t), u(t))$ enačbe $r' = F$. Zapiši karakteristični sistem 
        \item Pokaži, da iz tega karakterističnega sistema sledi karakterističi sistem za kvazlinearno enačbo
        \item Kako so geometrijsko prikazane normale Mongeejevega stolpca v kvazlinearnem sistemu?
        \item Kaj sta $p$ in $q$?
        \item Kaj moramo narediti, da dobimo dobro definiran sistem?
        \item Geometrično opiši kaj bomo naredili
        \item Sistem kakšne oblike želimo?
        \item Zakaj?
        \item Pokaži, da je konstrukcija traku smiselna
        \item Odvajaj začetno PDE po $x$ in po $y$
        \item Katera odvoda $p$ in $q$ sta enaka? Zakaj? 
        \item Vstavi v odvajano PDE
        \item Čemu sta enaka odvoda $p$ in $q$ po $t$?
        \item Združi vse v karakteristični sistem 5 enačb
        \item Poimenuj pogoja katerima mora ustrezati začetna krivulja?
        \item Definiraj projekcijo iz $R^5$ na krivuljo v $R^3$
        \item Navedi kompatibilnostna pogoja
        \item Navedi tranzverzalnostni pogoj
    \end{enumerate}

    \subsection{Eksistenčni izrek za PDE 1. reda}
    \begin{enumerate}
        \item Navedi izrek. Kaj so pogoji? Kaj je Bistvo izreka?
        \item Katera preslikava podaja parametrizacijo ploskve?
        \item Kaj je rešitev?
        \item Kaj velja za rešitev?
        \item Kako dobimo izražave $u$, $p$ in $q$ v odvisnosti od $x$ in $y$?
        \item Kakšno rešitev da kompatibilnostni pogoj?
        \item Kaj zagitavlja izpolnjevanje kompatibilnostnih pogojev?
        \item Kaj iščemo?
        \item TRDITEV - Navedi trditev, ki nam to zagotavlja 
        Dokaz
        \item Kaj označimo z $A_1$ in $A_2$?
        \item Definiraj preslikavo $F$
        \item Kaj iščemo? 
        \item Kaj že vemo?
        \item Kateri izrek uporabimo?
        \item Kaj je njegov pogoj?
        \item Kaj sledi?
    \end{enumerate}

    \section{Linearne PDE 2. reda}
    \begin{enumerate}
        \item Navedi 2 razloga zakaj so PDE pomembne
        \item Kakšne oblike je linearna PDE 2. reda? Kaj je na levi in kaj na desni strani?
        \item Kakšen predpis ima linearni operator?
        \item Kakšna je linearna PDE 2. reda za dve neznanki?
        \item Definiraj glavni del operatorja $L$
        \item Klasificiraj enačbe 2. reda dveh spremenljivk. Za vsako podaj pogoj
        \begin{itemize}
            \item Hiperbolična
            \item Parabolična
            \item Eliptična
        \end{itemize}
        \item Kaj pomeni, da je klasifikacija smiselna?
        \item Kako dobimo vektor novih spremenljivk?
        \item Kako se izraža rešitev v novih spremenljivkah?
        \item Izračunaj odvode, ki nastopajo v PDE 2. reda za dve spremenljivki
        \item Kaj je glavni del operatorja $L$ v starih spremenljivkah?
        \item Kaj je glavni del operatorja $L$ v novih spremenljivkah?
        \item Izrazi koeficiente v glavnem delu operatorja v novih spremenljivkah s koeficienti v glavnem delu operatorja v starih spremenljivkah
        \item Kaj predstavljajo te izražave?
        \item Zapiši kvadratno formo v matrični obliki
        \item Kaj velja za determinanto?
        \item Kaj zato velja za tip enačbe?
    \end{enumerate}

    \section{Kanonične oblike}
    \begin{enumerate}
        \item Kaj je finta kanonične oblike?
        \item Kaj je kanonična oblika hiperbolične enačbe?
        \item Kaj je kanonična oblika eliptične enačbe?
        \item Kaj je kanonična oblika parabolične enačbe?
    \end{enumerate}

    \subsection{Kanonična oblika hiperboličnega tipa}
    \begin{enumerate}
        \item Kaj pravi izrek o kanonični obliki hiperboličnega tipa?
        Dokaz
        \item Kako se glasi naša enačba v novih spremenljivkah in kaj želimo?
        \item Kako izrazimo željeno s starimi spremenljivkami?
        \item Kaj predstavljata dobljeni enačbi?
        \item Kako rešujemo tako enačbo in kaj dobimo?
        \item Zapiši enačbo v drugačni obliki
        \item Kaj iščemo in kako to izračunamo?
        \item Kaj sta dobljeni enačbi?
        \item Kako rešujemo dobljeni enačbi?
        \item Kaj je rešitev?
        \item Kako se izrazi $\xi$?
        \item Kaj je geometrični pomen te izražave?
        \item Kako dobimo drugo kanonično spremenljivko? $\zeta$?
        \item Navedi Triconijevo enačbo
        \item Kdaj je enačba hiperbolična?
        \item Kaj je njen kanonični sistem?
        \item Poišči njene kanonične koordinate
        \item Zapiši enačbo v kanoničnih koordinatah
    \end{enumerate}

    \subsection{Kanonična oblika paraboličnega tipa}
    \begin{enumerate}
        \item Kaj pravi izrek o kanonični obliki paraboličnega tipa?
        Dokaz
        \item Kaj smemo predpostaviti po definiciji enačbe paraboličnega tipa? Zakaj?
        \item Kaj moramo pokazati?
        \item Dovolj je pokazati le eno enakost. Zakaj?
        \item Pretvori enačbo na lažjo obliko
        \item Kako izberemo kanonične koordinate?
        \item PRIMER - Najdi kanonične koordinate enačbe $x^2 * u_{xx} - 2xy * u_{xy} + y^2 * u_{yy} + x * u_x + y * u_y = 0$
    \end{enumerate}

    \subsection{Kanonična oblika eliptičnega tipa}
    \begin{enumerate}
        \item Kaj pravi izrek o kanonični obliki eliptičnega tipa?
        \item Za kakšne koeficiente originalne enačbe iščemo kanonične koordinate?
        Dokaz
        \item Kakšna je enačba v novih koordinatah in kaj velja za njene koeficiente?
        \item Izrazi nove koeficiente s starimi
        \item Uporabi 3. vprašanje
        \item Definiraj novo funkcijo
        \item Kateri izraz je ekvavilenten sistemu iz 5. vprašanja
        \item Za katero funkcijo in izraz velja enako?
        \item Kako dobimo enačbi prvega reda za ti dve na novo definirani preslikavi?
        \item Kateri enačbi dobimo?
        \item Kako dobimo $\Phi(x, y)$ in $\Psi(x ,y)$?
        \item Kako sta $\Phi(x, y)$ in $\Psi(x ,y)$ povezana z rešitvjo $u$?
        \item Čemu je ekvavilentna prvotna enačba?
        \item Kdaj ima definiranje $\Phi(x, y)$ in $\Psi(x ,y)$ smisel?
        \item Izrazi koorfinati $\xi$ in $\zeta$
        \item Kako iz starih koeficientov PDE dobimo nove? Zapiši v matrini obliki
        \item Uporabi to v našem primeru in izračunaj kanonično obliko enačbe
        \item Kaj je translacija pri hiperboličnem tipu?
        \item Kako jo dobimo?
        \item Kako se imenujejo translacije?
    \end{enumerate}

    \section{Podaj primer hiperbolične enačbe}
    \begin{enumerate}
        \item Kaj je njena formula?
        \item Domena
        \item Kanonična oblika?
        \item Kanonični spremenljivki?
        \begin{itemize}
            \item Kaj je karakteristični sistem?
        \end{itemize}
        \item Kaj je kanonična oblika?   
        \item Reši kanonično enačbo
        \item Kaj je originalnih koordinatah?
        \item Definiraj krepko rešitev
        \item Definiraj šibko rešitev?
        \item Je limita še vedno 2-krat odvedljiva?
        \item Kaj so karakteristične premice valovne enačbe?
        \item Kakšna informacija se prenaša vzdolž karakteristik neke hiperbolične enačbe?
        \item Zapiši valovno enačbo na drugačen način
        \item V kakšni zvezi sta dobljena člena?
        \item Kaj velja za rešitev?
        \item Kaj sta rešitvi "manjših" PDE?
        \item Kako se imenujeta rešitvi?
        \item Kako rečemo rešitvi $u = F + G$?
        \item Kako se glasi Cauchyjeva naloga valovne enačbe?
        \item Izrazi $F$ in $G$ z začetnimi pogoji
        \item Kaj je torej rešitev?
        \item Kako se imenuje?
        \item IZREK - Podaj eksistenčni izrek za valovno enačbo
        \item Kaj pomeni dobra pogojenost?
        Dokaz
        \item Kako dokažemo obstoj in enoličnost?
        \item Kdaj je rešitev dvakrat zvezno odvedljiva?
        \item Kaj pomeni zvezna odvisnost?
        \item Pokaži, da je res zvezno odvisna od začetnih pogojev
    \end{enumerate}

    \section{Nehomogena valovna enačba}
    \begin{enumerate}
        \item Cauchyjev problem nehomogene valovne enačbe
        \item Kaj v homogeni enačbi vpliva na vrednosti njene rešitve?
        \item Kje je definirana rešitev homogene?
        \item Kaj je karakteristični trikotnik?
        \item Kje je definirana rešitev nehomogene?
        \item TRDITEV - Koliko rešitev ima nehomogen problem?
        \item Dokaz
        \begin{itemize}
            \item Definiraj novo funkcijo in zapiši katero enačbo rešitev
            \item Kaj so začetni pogoji te funkcije?
            \item Po čem sledi enoličnost rešitve nehomogene?
        \end{itemize}
        \item IZREK - Greenov izrek (formula)
        \item Izračunaj drugi integral nehomogenega dela po karakterističnem trikotniku
        \begin{itemize}
            \item Uporabi greenovo formulo
            \item Izračun po bazi
            \item Izračun po desni strani
            \item Izračun po levi strani
            \item Združi v eno vrstico
            \item Kaj je dobljena rešitev?
        \end{itemize}
    \end{enumerate}

    \subsection{Duhamelov postopek}
    \begin{enumerate}
        \item Kaj je finta postopka?
        \item TRDITEV - S pomočjo rešitve katerega problema znamo rešiti našo valovno enačbo pri pogojih $u = 0$ in $u_t = 0$?
        Dokaz
        \item Označi integral pomožne funkcije
        \item Kaj so začetni pogoji te funkcije?
        \item Katero PDE porodi?
        \item Kaj je rešitev za $v$?
        \item Kaj je rešitev zaćetnega problema za $u$?
        \item Kateri splošnejši problem želimo po navadi rešiti?
        \item Kaj je njegova rešitev?
        \item Zapiši formulo rešitve
    \end{enumerate}

    \section{Uporaba Fourierove analize pri reševanju PDE}
    \begin{enumerate}
        \item Kaj je nastavek za rešitev?
        \item Kako lahko opišemo našo rešitev?
        \item Navedi toplotno enačbo na končnem nosilcu
        \item Kje je definirana rešitev?
        \item Navedi Dirichletov robni pogoj
        \item Kaj je začetni pogoj
        \item Ob skici pojasni kaj imamo podano in kaj želimo zračunati
        \item S kakšnim nastavkom rešujemo enačbo?
        \item Prevedi toplotno enačbo na sistem enačb
        \item Kako se imenuje sistem?
        \item Kaj je prostorski del našega problema? Kaj je robni pogoj?
        \item Katere tri možnosti ločimo?
        \item Kaj je splošna rešitev prostorskega dela pri $\lambda < 0$?
        \item Kaj je rešitev?
        \item Kaj je splošna rešitev prostorskega dela pri $\lambda = 0$?
        \item Kaj je rešitev?
        \item Kaj je splošna rešitev prostorskega dela pri $\lambda > 0$?
        \item Kaj je rešitev?
        \item Kaj smo reševali?
        \item Kaj so lastne vrednosti?
        \item Kaj so lastni vektorji?
        \item Kaj so rešitve časovnega dela?
        \item Kaj je rešitev in zakaj?
        \item Katero rešitev moramo poiskati?
        \item Kako razvijemo začetni problem v vrsto?
        \item Kaj so koeficienti v razvoju začetnega problema?
        \item Kaj je torej rešitev?
        \item Kaj pomeni, da je rešitev krepka?
        \item S čem lahko dokažemo krepkost?
        \item Kaj je Neumannov pogoj za začetno enačbo?
    \end{enumerate}

    \subsection{Neomogeni začetno-robni problemi}
    \begin{enumerate}
        \item Navedi sistem za nehomogen začetno robni problem valovne enačbe
        \item Ali lahko rešujemo s seperacijo spremenljivk?
        \item Kako lahko opazujemo homogen sistem?
        \begin{itemize}
            \item Kot kakšno funkcijo smo opazovali rešitev problema?
            \item Kako lahko zapišemo krivulje v $R^3$?
            \item Kako zapišemo krivuljo v $R^3$ glede na neko bazo?
            \item Kaj so baze v prostoru spremenljivke $x$?
            \item Kako lahko zapišemo rešitev glede na bazo?
        \end{itemize}
        \item S kakšnim nastavkom lahko rešujemo nehomogeno enačbo?
        \item Kako?
        \item Čemu ustreza desna stran enačbe?
        \item Najdi koeficiente v razvoju rešitve odvisne od $t$
        \item Kako določimo neznane koeficiente?
        \item Enačbe kakšne oblike smo reševali?
        \item Kaj je $\mathcal{L}$
        \item Kaj je definicijsko območje, zaloga prednosti in predpis za $\mathcal{L}$
        \item Kaj smo dobili?
        \item Kaj pomeni KONS?
        \item Kaj predpostavljamo za lastne vektorje?
        \item Kako lahko zato izrazimo funkcijo definirano na prostosu od $x$?
        \item Kako lahko predstavimo rešitev?
        \item Vstavi rešitev v toplotno reačbo
        \item Kaj dobimo?
        \item Kdaj znamo rešiti naš problem?
        \item Kaj je Sturm-Lionvilleeva teorija?
    \end{enumerate}

    \subsection{Toplotna enačba na neskončnem nosilcu}
    \begin{enumerate}
        \item Katero metodo uporabljamo pri tem?
        \item Kako lahko razvijemp periodično funkcijo v vrsto?
        \item Kaj je baza? Na katerem prostoru?
        \item Kaj so koeficienti v vrsti?
        \item Vpelji novo oznako funkcije
        \item Vpelji oznako za spremenljivko
        \item Kaj je razlika med dvema zaporednima spremenljivkama?
        \item Čemu je enaka začetna funkcija?
        \item Kako se imenuje vsota, ki jo dobimo?
        \item Zapiši $f$ z integralom
        \item Definiraj Fourierovo transformacijo
        \item Definiraj inverzno Fourierovo transformacijo
        \item Definiraj Schwarzov razred
        \item Kaj velja za Fourierovo transformacijo? (2)
        \item Kakšen začetni problem želimo rešiti s Fourierovo transformacijo?
        \item Kako dobimo integralsko jedro, če je začetni problem definiran na Schwarzovem razredu?
    \end{enumerate}

    \section{Diracova delta "funkcija"}
    \begin{enumerate}
        \item Kako generiramo Diracovo delta funkcijo? 
        \begin{itemize}
            \item Kaj je nosilec?
            \item integral
            \item (pol)
        \end{itemize}
        \item Opis $\delta_0 (x)$
        \item Kako bi to intuitivno opisal? 
        \item Nariši vse
        \item Kako pridemo do evaluacije?
        \begin{itemize}
            \item integral
            \item Kaj velja za  $\delta_0 (x)$?
            \item Zakaj to velja?
            \item Oznaka evaluacije
        \end{itemize}
        \item Kaj v resnici definicija evaluacije?
        \item Kako imenujemo  $\delta_0 (x)$?
        \item Kako bi laho zapisali ealuacijo kot skalarni produkt?
        \item Po kateri mertiki?
        \item Podaj primer
        \begin{itemize}
            \item Kaj je njena *posebnost*
        \end{itemize}
        \item Definiraj prostor testnih funkcij
        \item Definiraj prostor distirbucij
        \item Prostor česa je torej to? Kaj je to?
        \item Podaj primer elementa iz prostora 
        \item Podaj primer - REGULARNA DISTRIBUCIJA
        \item Definiraj odvod distribucije 
        \item Kaj je odvod regularne distribucije? 
        \item Definiraj odvod  $\delta_0 (x)$
        \item Definiraj  $\delta_0 (x)$ s polom v poljubni točki
    \end{enumerate}

    \section{Integralska jedra}
    \begin{enumerate}
        \item Kje do zdaj se skriva linearen operator?
        \item Kakšen linearen operator vzamemo?
        \item Definiraj jedro
        \item Po kakšnem integralu integriramo?
        \item Kaj je disktretna verzija integralskega jedra?
        \item Zapiši ekvavilentna matrična zapisa (integralskega in matričnega)
        \item TRDITEV - Je Diracova funkcija soda ali liha? 
        \item Dokaži
        \item Kaj sledi iz tega? Pokaži, da res
    \end{enumerate}

    \subsection{Iskanje toplotnega jedra}
    \begin{enumerate}
        \item Kaj je toplotna enačba?
        \item S kakšnim nastavkom jo rešujemo?
        \item Vstavi v enačbo in posploši
        \item Zakaj lahko notranjost integrala enačimo z 0?
        \item Izrazi koeficiente nastavka
        \item Kako dobimo konstanto, ki nastopa v koeficientu iz nastavka?
        \item Izračunaj jo
        \item Kaj je torej rešitev?
        \item Kaj je najenostavnejši začetni pogoj za to rešitev?
        \item Izračunaj rešitev
        \item Polepšaj zapis rešitve
        \item Kaj predstavlja rešitev?
        \item TRDITEV - Kaj je rešitev $u(x - a, t)$? Dokaz
        \item Kaj je rešitev začetnega problema, če je začetni pogoj poljubna delta funkcija?
        \item TRDITEV - Kaj je posledica linearnosti toplotne enačbe? Dokaz 
        
        Rešitev splošnega začetnega problema 
        \item Kako lahko drugače zapišemo $f(x)?$
        \item Kaj je torej rešitev?
        \item Kako bi opisala kaj smo dobili?
        \item Kaj v tem je toplotno jedro?
    \end{enumerate}

    \section{Eliptična enačba}
    \begin{enumerate}
        \item S kakšnimi enačbami se bomo ukvarjali? (2)
        \item Poimenuj ju
        \item Kako je definiran Laplacov operator?
        \item Kaj pa v več dimenzijah?
        \item Kaj opisujeta enačbi?
        \item Kaj je enačba nehomogene toplotne enačbe v dveh spremenljivkah?
        \item Kje je podan robni pogoj?
        \item Kaj pomeni, da je problem stacionaren (pri velikih časih)?
        \item Kako opišemo stacionarno stanje?
        \item Kaj je enačba eliptičnega začetnega problema? 
        \item Naštej tri vrste robnih problemov
        \item Opiši Dirichletov problem
        \item Opiši Neumannov problem
        \item Kaj je Neumannov odvod?
        \item Ali so lahko podatki Neumannovega problema povsem poljubni?
        \item Opiši mešani problem
        \item Kdaj je Neumannov problem rešljiv?
        \item Iz katerega zreka to sledi?
        \item Navedi ta izrek
        \item V kaj pretvorimo izrek v našem primeru?
        \item Kaj mora torej veljati v Neumannovem primeru?
        \item Kaj mora veljati v Dirichletovem primeru?
        \item Definiraj harmonično funkcijo
        \item Kaj velja za vse harnomične funkcije?
        \item Definiraj rotacijsko simetričnost. Kaj to intuitivno pomeni?
        \item Zapiši laplacov operatov v polarnih koordinatah
        \item Kako se torej glasi laplacova enačba?
        \item Kaj pa če je $W$ rotacijsko simetrična?
        \item Kaj je rešitev?
        \item Kako se rešitev imenuje?
    \end{enumerate}

    \subsection{Princip maksima}
    \begin{enumerate}
        \item IZREK - krepki princip maksima
        \item IZREK - šibki princip maksima
        \item IZREK - princip sfecičnih povprečij
        \item Dokaz principa sferičnih povprečij
        \begin{itemize}
            \item Vpelji funkcijo
            \item Kaj hočemo pokazat?
            \item Kaj bo iz tega sledilo?
            \item Dokaz 1
            \item Dokaz 2
        \end{itemize}
        \item Kaj je obrat izreka sferičnih povprečij?
        \item Dokaz obrata
        \begin{itemize}
            \item Predpostavi, da ni harmonična. Kje?
            \item Na operatorju (integral) uporabi Greenovo formulo
            \item S čem je dobljeno v protislovju?
        \end{itemize}
        \item Navedi krepki princip maksima
        \item Dokaz
        \begin{itemize}
            \item Nariši skico
            \item Kaj hočemo dokazat?
            \item Predpostavi nasprotno - dobimo dve možnosti
            \item Katera možnost takoj odpove? Zakaj?
            \item Zakaj druga možnost ne velja?
            \item Kako dobimo, da je u konstantna na celotnem območju?
        \end{itemize}
    \end{enumerate}

    \subsection{Primeri uporabe principa maksima}
    \begin{enumerate}
        \item IZREK - Kdaj ima Dirichletov problem največ eno rešitev?
        \item Dokaz
        \begin{itemize}
            \item Def novo funkcijo
            \item Je harmonična? Zakaj?
            \item Koliko sta njen min in max?
        \end{itemize}
        \item Kaj pa če območje ni omejeno? Podaj primer z več kot eno rešitvijo
        \item IZREK - od česa je odvisna rešitev?
        \item Dokaz
        \begin{itemize}
            \item Definiraj dva problema in novo funkcijo z njima
            \item Kaj je problem za novo funkcijo?
            \item Kaj naj velja za robna pogoja?
            \item Kaj potem sledi za novo funkcijo?
        \end{itemize}
    \end{enumerate}

    \section{Osnove Greenove teorije}
    \begin{enumerate}
        \item Pri čem se uporablja?
        \item Katere probleme rešuje?
        \item Kako jih rešuje?
        \item Kaj iščemo v Dirichletovem primeru? Kje sta definirani?
        \item Kaj mora zanju veljati?
    \end{enumerate}

    \subsection{Greenove identitete}
    \begin{enumerate}
        \item Iz česa sledijo greenove identitete?
        \item Kaj je 1. Greenova identiteta?
        \item Kaj je 2. Greenova identiteta?
        \item Kaj je 3. Greenova identiteta?
        \item IZREK - Posledica česa je?
        \begin{itemize}
            \item Kaj so pogoji?
            \item Koliko rešitev ima Dirichletov problem?
            \item Koliko rešitev ima mešani problem?
            \item Koliko rešitev ima Neumannov problem in kakšne oblike so?
        \end{itemize}
        \item Dokaz za Dirichletov
        \item Dokaz za mešani
        \begin{itemize}
            \item Definiraj novo funkcijo
            \item Kaj je njen začetni problem?
            \item Kaj dobimo z 2. Greenovo identiteto?
            \item Kaj sledi za normalo gradienta nove funkcije?
            \item Kaj sledi za to novo funkcijo?
        \end{itemize}
        \item DOkat za Neumannov problem
        \begin{itemize}
            \item Kaj hočemo dokazat?
            \item Definiraj novo funkcijo
            \item Kaj je njen začetni problem?
            \item Vstavi v 2. Greenovo identiteto
            \item Kaj velja za gradient nove funkcije?
            \item Kakšna je torej nova funkcija?
        \end{itemize}
    \end{enumerate}

    \subsection{Greenova funkcija}
    \begin{enumerate}
        \item Kaj je motivacija?
        DIRICHLETOV PROBLEM
        \item Kaj je rešitev fundamentalna Laplacove enačbe?
        \item Kaj je rešitev Laplacove enačbe, ki je definirana povsod razen v eni točki
        \item Kakšna je ta rešitev?
        \item Kaj pa velja za $u$?
        \item Vpelji oznako rešitve
        \item Kako bomo iskali jedro?
        \item Kakšna mora biti $u$ in kakšen $\Omega$?
        \item Kje lahko zapišemo 3. Greenovo formujo za $u$? Formalno zapiši to območje
        \item Uporabi 3. Greenovo identiteto za $u$ in $\Gamma$ na tem območju
        \item Uporabi vse kar veš o območjih in funkcijah, ki nastopajo v tej enačbi
        KAJ DOBIMO KO GRE $\epsilon \Rightarrow 0$?
        \item Kaj je fundamentalna rešitev v polarnih koordinatah?
        \item Kaj je rešitev za točke iz roba kroga, ki smo ga odvzeli?
        \item Kako se glasi prvi člen - integral po robu kroga
        \item Kaj dobimo v limiti?
        \item Kako se glasi drugi člen - integral po robu kroga?
        \item Kaj dobimo v limiti?
        \item Kaj je torej rešitev?
        \item Kako se imenuje rešitev?
        \item Kakšna bo torej rešitev za naš (Dirichletov) problem?
        \item Kako bomo modificirali funkcije, da bomo rešitev izrazili z že znanimi funkcijami?
        \item Koliko funkcij rabimo?
        \item Definiraj jih
        \item Definiraj preslikavo, ki jo bomo uporabili v rešitvi
        \item Zapiši Greenovo reprezentijsko formulo za $u$, $h$ in $\Gamma$
        \item Izrazi rešitev $u$ z $G_{\Omega}$
        \item IZREK - Povzetek
        \item Navedi Greenovo reprezentijsko formulo s funkijo $G_{\Omega}$ 
        \item Kako dokažemo?
    \end{enumerate}
    
    \subsection{Fundamentalne rešitve}
    \begin{enumerate}
        \item Kako se glasi Greenova reprezentijska formula za testno funkcijo?
        \item Navedi nekoliko splošnejšo obliko
        \item Razpiši
        \item Kako integriramo naprej?
        \item Kaj potem dobimo?
        \item Kako se dobljen rezultat vede?
        \item Čemu je torej ta integral enak?
        \item Kaj pa velja za $grad(\Gamma)$?
        \item Definiraj fundamentalno rešitev Laplacove enačbe
        \item Navedi še boljšo definicijo fundamentalne rešitve
        \item Koliko je fundamentalnih rešitev? Dokaži
        \item Podaj primer fundamentalne rešitve
        \item Definiraj adjungirana operatorja
        \item Podaj primer sebi adjungiranega operatorja (2x) 
        \item Podaj primer operatorja, ki ni samemu sebi adjungiran
        \item Definiraj fundamentalno rešitev z adjungiranim operatorjem
        \item Čemu je enak parcialni diferencialni operator uporabljen na fundamentalni rešitvi?
    \end{enumerate}

    \section{Neumannova funkcija}
    \begin{enumerate}
        \item Kaj je motivacija?
        \item Pokaži, da Neumanove funkcije ne moremo dobiti na isti način kot Greenove
        \begin{itemize}
            \item Kaj je Greenova reprezentijska formula?
            \item S čem smo nadomestili $\Gamma$?
            \item Uporabi kompatibilnostni pogoj
            \item Čemu je torej enaka rešitev?
            \item Kaj dobimo, če je $u = 1$?
        \end{itemize}
        \item Kako popravimo robni pogoj za $h$?
        \item Definiraj Neumannovo funkcijo
        \item Kaj je rešitev Neumannovega problema?
        \item Zakaj se rešitve razlikujejo le za konstanto?
        \item Poišči dve rešitvi Neumannovega problema
    \end{enumerate}

    \section{Greenova funkcija in konformne preslikave}
    \begin{enumerate}
        \item Kaj je motivacija?
        \item Definiraj konformno preslikavo
        \item Definiraj kompleksno odvedljivost
        \item Kako razdelimo kompleksno odvedljivo funkcjo?
        \item Definiraj odvod te razdelitve
        \item Kaj je matrika odvoda?
        \item Zapiši "realno" definicijo odvoda
        \item Zapiši jo v realni obliki
        \item Zapiši sistem iz odvodov $\phi$
        \item Kaj sledi iz tega za realni in imaginarni del funkcije?
        \item IZREK - Med kakšnimi območji obstajajo konformne preslikave?
        \item V katero območje lahko vedno slikamo? aka podaj primer takega območja
        \item IZREK - Definiraj preslikavo in zapiši Greenovo funkcijo za Dirichletov problem
        Dokaz
        \item Kaj velja zaradi konformnosti preslikave iz izreka?
        \item Kako jo lahko zapišemo drugače?
        \item Kaj velja za novo funkcijo v zapisu? Pokaži
        \item Ali je $\Psi(\xi, \xi) = 0$? Zakaj?
        \item Razpiši desno stran greenove funkcije
        \item Kako bi torej zapisali Greenovo funkcijo?
        \item Kaj velja za $h$ v tem zapisu? 
        \item Kaj še moramo dokazati? (2)
        \item 1
        \begin{itemize}
            \item Kaj sledi iz konformnosi?
            \item Kam se slika rob?
            \item Kakšna je torej preslikava med roboma?
            \item Koliko je Greenova funkcija enaka?
        \end{itemize}
        \item 2
        \begin{itemize}
            \item Kaj je $h$?
            \item Razpiši $\Psi$
            \item Kateri del $h$ oz. $\Psi$ mora biti harmoničen?
            \item Zakaj je $\Psi$ holomorfna?    
        \end{itemize}
        \item Kako iz konformne preslikave dobiš preslikavo iz izreka?
    \end{enumerate}

    \section{Valovna enačba v $R^3$}
    \begin{enumerate}
        \item Kaj je valovna enačba v 1+1?
        \item Kaj je rešitev valovne enačbe v 1+1?
        \item Kaj je rešitev, če je $f = 0$?
        \item Nariši skico vala 
        \item Vrednosti rešitve v 5 različnih točkah?
        \item Kako imenujemo valovno enačbo v $R^3$ 
        \item Kaj je valovna enačba v $R^3$?
        \item Kakšna morata biti začetna pogoja?
        \item Zapiši problem za valovno enačbo v sferičnih koordinatah
        \item Kakšne rešitve iščemo?
        \item Na kaj se pretvori valovna enačba?
        \item Kako jo pretvorimo na standardno valovno enačbo?
        \item Kaj želimo rešiti s pomočjo tega?
        \item Kaj je problem pri reševanju?
        \item Kako razširimo začetna pogoja?
        \item Kaj je torej rešitev?
        \item Kaj je splošen problem?
        \item LEMA - Kako sta povezani rešitvi problemov, ki imata ničeln $f$ oz. $g$?
        \item Dokaz
        \item Kako razdelimo problem valovne enačbe na dva podproblema?
        \item Definiraj sferično povprečje
        \item DARBOUXOV IZREK
        Dokaz
        \item Zakaj hočemo sferično povprečje zapisati v lepši obliki?
        \item Zapiši sferično povprečje v lepši obliki
        \item Koliko je odvod sferičnega povprečja po $a$?
        \item Kako vidimo, da res reši enačbo?
        \item Kateri začetni problem reši sferično povprečje?
        \item Kakšen operator je sferično povprečje?
        \item Koliko je sferično povprečje pri radiju 0?
        \item IZREK - Če u repi valovno enačbo v 3D, katero enačbo reši sferično povprečje za $u$?
        \item Dokaz
        \item Če $u$ reši valovno enačbo za $f = 0$, Katero enačbo reši $M_u$ in pri katerem začetnem pogoju?
        \item Kaj je tu parameter in kaj aktivna spremenljivka?
        \item Katero enačbo in pri katerem začetnem pogoju reši $M_u$ po izreku?
        \item Kaj je tu parameter in kaj aktivna spremenljivka?
        \item Kako razširimo na celo realno os?
        \item Kako se glasi rešitev?
        \item Kako pa dobimo rešitev $u$?
        \item Kako se glasi rešitev $u$?
        \item ?
    \end{enumerate}

\end{document}