\documentclass{article}
\usepackage{amsmath}
\usepackage{xcolor}
\title{Algebra 2}

\begin{document}
    \maketitle

    \section{Končne grupe}
    \subsection{Posledice Lagrangeeovega izreka}
    \begin{enumerate}
        \item Navedi Lagrangeov izrek
        \item Navedi vse majhne grupe do izomorfizma natančno do reda 8
        \item POSLEDICA 1 - Kakšno je razmerje med redom grupe in redom njenih podgrup?
        \item Za katere $m$ velja $a^m = 1$? Dokaži ekvavilenco
        \item Kaj velja za praštevilo $p$, če $a^p = 1$?
        \item Kakšen je red slike homomorfizma?
        \item Kdaj je red slike enak redu elementa
        \item Kakšen je red odseka $aN$?
        \item POSLEDICA 2 - Kakšni so redi elementov grupe? Dokaži
        \item POSLEDICA 3 - Kaj velja za vsak element grupe reda $n$? Dokaži
        \item POSLEDICA 4 - Navedi Fermatov mali izrek. Navedi še njegovo drugo obliko
        \begin{itemize}
            \item Dokaz
        \end{itemize} 
        \item POSLEDICA 5 - Kdaj je grupa ciklična? Kateri elementi jo generirajo?
        \item \begin{itemize}
            \item Dokaz
        \end{itemize} POSLEDICA 6 - Čemu je ekvivalentno dejstvo, da ima ciklična grupa praštevilski red? Dokaz
    \end{enumerate}

    \subsection{Razredna formula}
    \begin{enumerate}
        \item Definiraj kdaj sta elementa konjugirana
        \item Kakšna relacija je konjugiranost? Dokaži
        \item Definiraj konjugiranostni razred?
        \item Čemu je enaka vsota redov konjugiranostnih razredov?
        \item Kdaj ima konjugiranostni razred en sam element?
        \item Definiraj centralizator elementa $a$
        \item Kdaj je centralizator elementa enak celi grupi?
        \item LEMA - V kakšnem razmerju je centralizator do grupe? Dokaži
        \item LEMA - Navedi formulo, ki povezuje navedene pojme. Dokaži
        \item Kaj velja za element iz centra grupe? (2 lastnosti)
        \item Koliko je konjugiranostnih razredov z enim elementom?
        \item Navedi razredno formulo
    \end{enumerate}

    \subsection{Cauchyjev izrek}
    \begin{enumerate}
        \item Navedi Cauchyjev izrek
        \item Dokaz
        \begin{itemize}
            \item S katero metodo dokazujemo?
            \item Dokaži če je red grupe enak $p$
            \item Dokaži, če grupa ni Abelova in $n > p$
            \item Dokaži, če je grupa Abelova in $n > p$
        \end{itemize}
        \item Navedi Cauchyjev izrek malo drugače. Dokaži ekvavilenco
        \item Definiraj p-grupo
        \item Za kakšne grupe ima ta definicija smisel?
        \item Kaj je ekvivalentno temu, da je $G$ p-grupa?
    \end{enumerate}

    \subsection{Delovanja grup}
    \begin{enumerate}
        \item Definiraj delovanje grupe G na množici X
        \item Definiraj orbito elementa x
        \item Definiraj stabilizator elementa x
        \item PRIMER 1 - Navedi delovanje s katerim grupa deluje sama na sebi
        \item PRIMER 2 - Pokaži, da je konjungiranje delovanje
        \item PRIMER 3 - Navedi delovanje s katerim grupa deluje na množici vseh odsekov podgrupe H
        \item PRIMER 4 - Navedi delovanje s katerim grupa deluje na množici vseh podgrup
        \item PRIMER 5 - Navedi delovanje s katerim grupa $S_n$ deluje na množici vseh polinomov v n spremenljivkah
        \item PRIMER 6 - Navedi delovanje s katerim grupa $GL_n(R)$ deluje na množici $R^n$
        \item Kaj je trivialno delovanje?
        \item V kakšnem razmerju je stabilizator z dano grupo?
        \item Definiraj ekvavilenčno relacijo
        \item Kaj so njeni ekvavilenčni razredi?
        \item Navedi izrek o orbiti in stabilizatorju (formula). Navedi preslikavo, ki formulo dokazujemo
        \item Navedi razredno formulo v skladu z novimi oznakami
        \item Kaj velja za to formulo, če je G končna p-grupa?
    \end{enumerate}

    \subsection{Izreki Sylowa}
    \begin{enumerate}
        \item Definiraj normalizator
        \item Navedi tri trditve, ki veljajo za normalizator
        \item Definiraj p-podgrupo
        \item Definiraj p-podgrupo Sylowa
        \item (definiraj najprej red grupe G) Kakšen red imajo p-podgrupe?
        \item (definiraj najprej red grupe G) Kakšen red imajo p-podgrupe Sylowa?
        \item Za kakšno grupo G veljajo izreki Sylowa?
        \item IZREK 1 - Kdaj G vsebuje p-podgrupo reda $p^l$?
        \begin{itemize}
            \item S katero metodo dokazujemo ta izrek?
            \item Dokaži, če $p$ ne deli reda centra grupe
            \item Dokaži, če $p$ deli red centra grupe
        \end{itemize}
        \item IZREK 2 - V kakšnem razmerju so p-podgrupe in p-podgrupe Sylowa?
         \begin{itemize}
             \item Definiraj množico vseh odsekov
             \item Definiraj delovanje
             \item Definiraj množico $Z$
             \item Pokaži, da $Z$ ni prazna
             \item Pokaži, da je $H$ podgrupa
         \end{itemize}
         \item IZREK 3 - Iz dane p-podgrupe pridobi novo p-podgrupo. Zakaj to velja?
         \item POSLEDICA - Kdaj je p-podgrupa Sylowa podgrupa edinka?
         \item IZREK 4 - Kaj velja za število vseh p-podgrup?
         \begin{itemize}
             \item Definiraj množico vseh p-podgrup Sylowa
             \item Definiraj delovanje na tej množici
             \item Kaj je orbita?
             \item Kaj je stabilizator?
             \item Po lemi o orbiti in stabilizatorju privedi dokaz do konca
         \end{itemize}
         \item IZREK 5 - Kakšne oblike je število vseh p-podgrup?
         \begin{itemize}
             \item Definiraj množico vseh p-podgrup
             \item Definiraj delovanje na njej
             \item Definiraj podmnožico podgrup Sylowa $W$
             \item Kateri element je zagotovo vsebovan v $W$?
             \item Pokaži, da je to edini element, ki je vseboven v $W$
             \item Izpelji koliko je število p-podgrup Sylowa 
         \end{itemize}
    \end{enumerate}

    \subsection{Končne Abelove grupe}
    \begin{enumerate}
        \item Naštej vse končne abelova grupe
        \item LEMA 1 - Navedi lemo, s katero grupo razdelimo na dve podgrupi
        \item Dokaz
        \begin{itemize}
            \item kaj moraš dokazat?
            \item uporabi Lagrangeevo posledico
            \item uporabi tujust elementov
            \item pokaži, da je vsota direktna
        \end{itemize}
        \item Iz leme izpelji izomorfizem med cimličnimi grupami
        \item LEMA 2 - Posploši prvo lemo na splošnejše rede grupe G. Dokaz
        \item LEMA 3 - Kdaj je p-grupa ciklična? (ekvavilenca)
        \item Dokaz
        \begin{itemize}
            \item ( => ) če je G ciklična
            \item ( <= ) če je red(G) = p
            \item ( <= ) Kaj lahko poveš o edini podgrupi s p elementi?
            \item ( <= ) formalno zapiši kaj je ta podgrupa
            \item ( <= ) uporabi izrek o izomorfizmu in definiraj preslikavo
            \item ( <= ) kaj velja za sliko endomorfizma?
            \item ( <= ) kaj zato vemo o kvocientni grupi?
            \item ( <= ) kaj so elementi kvocientne grupe?
            \item ( <= ) kako lahko torej razpišemo G?
            \item ( <= ) pokaži, da je G ciklična
        \end{itemize}
        \item LEMA 4 - Kako podgrupo dopolnimo do grupe? Kakšna mora biti podgrupa, da to sploh lahko storimo?
        \item Dokaz 
        \begin{itemize}
            \item kako dopolnimo, če je G ciklična?
            \item če G ni ciklična, kaj je IP.?
            \item Kaj nam o G in C pove lema 3?
            \item Kaj velja za 
        \end{itemize}
        \item Navedi osnovni izrek o končnih abelovih grupah
        \item Kaj so ciklične grupe?
        \item Kako izberemo podgrupe, če je G netrivialna?
        \item Dokaži osnovni izrek
        \item Navedi ekvivalentno formulacijo izreka
        \item Kdaj so si direktne vsote ekvavilentne?
        \item IZREK - Klasifikacija končnih Abelovih grup
        \item Dokaz
        \item Navedi osnovni izrek o končno generiranih abelovih grupahni
    \end{enumerate}

    \section{Deljivost v komutativnih kolobarjih}
    \subsection{Glavni ideali}
    \begin{enumerate}
        \item Definiraj glavni ideal
        \item Razloži vsebovanost elementa a in ideala (a)
        \item Kako je generiran glavni ideal?
        \item Kdaj je ideal glavni?
        \item Definiraj desni ideal generiran z a. Kako ga označimo?
        \item Definiraj levi ideal generiran z a. Kako ga označimo?
        \item Kakšne oblike elementi so v dvostranskem idealu generiranem z a? Kako ga označimo?
        \item S kakšnimi kolobarji se ukvarjamo mi?
        \item Katera dva ideala sta vedno glavna? S čem sta generirana?
        \item Kdaj je (a)=K? Zakaj?
        \item Kdaj sta to edina ideala? Kaj iz tega sledi?
        \item Kaj so glavni ideali celih števil? S čem je generiran? Kaj zato velja?
        \item Kaj so glavni ideali kolobarja F[X]? S čem so generirani? Kaj velja za glavne ideali F[X]?
        \item Definiraj končno generiran ideal
        \item Kaj vse vsebuje končno generiran glavni ideal?
        \item Elementi kakšne oblike so v končno generiranem idealu?
        \item Kako je končno generiran ideal povezan z glavnimi ideali?
        \item Kaj vsebuje ideal (4, 6) kolobarja celih števil? Čemu je enak?
        \item Kaj vsebuje ideal (2, X) kolobarja F[X]? Ali je glavni?
    \end{enumerate}

    \subsection{Deljivost in nerazcepnost}
    \begin{enumerate}
        \item Definiraj, da b deli a. Kakšen mora biti kolobar? Kako označimo?
        \item Poimenuj a in b
        \item Definiraj deljivost z glavnimi deali
        \item Dokaži ekvavilenco definicij deljivosti 
        \item Definiraj kdaj sta elementa asociirana
        \item Kaj velja za deliteje asociiranih elementov. Katere elemente asociirana elementa delita?
        \item Natanko kdaj sta si elementa celega kolobarja asociirana?
        \begin{itemize}
            \item dokaz (=>)
            \item dokaz (<=)
        \end{itemize}
        \item Kaj so asociirani elementi v kolobarju celih števil?
        \item Kaj so asociirani elementi v F[X]?
        \item Definiraj največji skupni delitelj elementov a in b. V kakšnem kolobarju je definiran?
        \item Definiraj kdaj sta elementa tuja
        \item Kaj je z enoličnostjo in obstojem največjega skupnega delitelja?
        \item Kaj velja za različne največje skupne delitelje?
        \item Kako dosežemo enoličnost v Z?
        \item Kako dosežemo enoličnost v F[X]?
        \item TRDITEV - Kdaj obstaja največji skupni delitelj kolobarja? Kakšne oblike je? Kakšen mora biti kolobar?
        \begin{itemize}
            \item dokaži, da obstaja skupni delitelj in njegovo obliko
            \item dokaži, da je največji
        \end{itemize}
        \item Definiraj nerazcepen element v kakem kolobarju ga definiramo?
        \item Definiraj razcepen element
        \item Kaj so nerazcepni elementi kolobarja celih števil?
        \item Najdi nek nerazcepen element v Z[X]. Zakaj ta element ni nerazcepen v Q[X]?
        \item Kaj so obrnljivi elementi v Z[i]? Najdi nek razcepen in nek nerazcepen element v tem kolobarju in razloži zakaj je razcepen oz. nerazcepen
        \item TRDITEV - Kdaj je element celega kolobarja nerazcepen (ekvavilenca)? Dokaži
    \end{enumerate}

    \subsection{Evklidski kolobarji}
    \begin{enumerate}
        \item Navedi izrek o deljenju za polinome
        \item Dokaz
        \begin{itemize}
            \item Dokaži, če je deljenec ničeln. Kaj vzamemo za $q(x)$ in kaj za $r(x)$?
            \item S katero metodo dokazujemo?
            \item Kaj vzamemo za $q(x)$ in kaj za $r(x)$ pri $st(f(x)) = 0$?
            \item Dokaži za $m -> m+1$
        \end{itemize}
        \item Definiraj evklidski kolobar
        \item Naštej tri osnovne evklidske kolobarje
        \item Najdi enega, ki ni evklidski
        \item Za vsakega od njih najdi preslikavo $\delta$ ($Z$, $F[X]$)
        \item Katera algebraična struktura je vedno evklidski kolobar? Kaj je njena preslikava $\delta$?
        \item IZREK - Kakšni so ideali evklidskega kolobarja? Dokaži
        \item POSLEDICA - Navedi ekvavilence temu, da je $p$ nerazcepen. Pri kakšnih pogojih to velja? Dokaži
        \item POSLEDICA - Kaj Zagotovo obstaja v evklidskem kolobarju?
        \item POSLEDICA - Zaradi katere se vpelje praelement. Dokaži
        \item Definiraj praelement
        \item Kaj velja za praelemente? (Od sošolca zapiski)
        \item Kaj velja za praelemente v Evklidskih kolobarjih?  Dokaži (Od sošolca zapiski)
        \item LEMA - Kdaj lahko tvorimo zaporedje elementov, katerih ideali so strogo naraščajoči?
        \item Navedi dve definiciji Noetherskega kolobarja
        \item Navedi Hilertov izrek o bazi
        \item Naštej tri kolobarje, ki so Noetherski
        \item Navedi nek kolobar, ki ni Noetherski
        \item LEMA - Kateri kolobarji so vedno Noetherski? Dokaži
        \item Kaj povesta zadnji dve lemi?
        \item Kaj so ideali Noetherskega kolobarja?
        \item Definiraj kolobar z enolično faktorizacijo
        \item Kaj pomeni "do asociiranosti natančno"?
        \item Kateri kolobarji imajo enolično faktorizacijo?
        \item Povzemi kaj velja za evklidske kolobarje
        \item Navedi osnovni izrek aritmetike za evklidske kolobarje
    \end{enumerate}

    \subsection{Nerazcepni polinomi}
    \begin{enumerate}
        \item Kdaj je kolobar polinomov evklidski?
        \item Kakšni so ideali kolobarja polinomov? Kakšne oblike so? Kaj jih generira?
        \item Definiraj največji skupni delitelj polinomov. Ali obstaja? Kakšne oblike je? Kdaj je enoličen?
        \item Kaj so obrnljivi elementi v kolobarju polinomov?
        \item Kateri elementi kolobarja polinomov so nerazcepni?
        \item Čemu je nerazcepnost polinomov ekvavilentna?
        \item Kako lahko zapišemo nekonstantne polinome?
        \item Kateri polinomi so nerazepni?
        \item Kaj pravi osnovni izrek algebre v $C$?
        \item Kateri polinomi so nerazcepni v $C$?
        \item Kako lahko zapišemo nekonstantne polinome v $C$?
        \item Kateri polinomi so nerazcepni v $R$?
        \item Kako lahko zapišemo nekonstantne polinome v $R$?
        \item TRDITEV - Kdaj ima polinom ničlo a? (kaj ga mora deliti?) Dokaži
        \item POSLEDICA - Kdaj so polinomi katere stopnje nerazcepni? 
        \item Kaj NE vpliva na nerazcepnost polinomov?
        \item Definiraj primitiven polinom na dva načina
        \item LEMA - Kaj je produkt primitivnih polinomov?
        \begin{itemize}
            \item S katero tehniko dokazujemo?
            \item Definiraj ideal v katerem je produkt polinomov
            \item Kaj pa sledi za ta ideal, ker sta $f(x)$ in $g(x)$ primitivna?
            \item Definiraj odseka
            \item Koliko je produkt teh odsekov?
            \item Kaj iz tega sledi za kvocientni kolobar?
            \item Privedi do protislovja z zadnjo točko
        \end{itemize}
        \item IZREK - Kako je z nerazcepnostjo polinomov v $Z[X]$ in $Q[X]$?
        \begin{itemize}
            \item Kaj moramo dokazati?
            \item Zapiši enakost
            \item Vpelji največje skupne delitelje in zapiši enakost
            \item Uporabi lemo
            \item Izpelji do konca
        \end{itemize}
        \item Ime leme, ki sledi iz leme in izreka
        \item Kaj pravi Eisensteinov kriterij?
        \begin{itemize}
            \item S katero tehniko dokazujemo?
            \item Kaj sledi iz izreka?
            \item Kaj velja za konstanten člen?
            \item Kaj velja za vodilni člen?
            \item Kaj velja za koeficiente enega od faktorjev (polinomov)?
            \item Razpiši koeficient začetnega polinoma
            \item Privedi do protislovja
        \end{itemize}
        \item Katera znana polinomska funkcija je nerazcepna?
        \item Definiraj množico primitivnih korenov
        \item Definiraj ciklonomične polinome
    \end{enumerate}

    \section{Ničle polinomov in razširitve polj}
    \subsection{Algebraični in transcedentni elementi}
    \begin{enumerate}
        \item Zakaj vpeljujemo nova polja?
        \item Naštej 7 primerov polj
        \item Navedi en algebraičen in en transcedentni element
        \item Kaj je razlika med transcedentnim in algebraičnim elementom?
        \item Definiraj algebraičen element
        \item Definiraj transcedentni element
        \item Definiraj minimalni polinom algebraičnega elementa $a$
        \item Definiraj stopnjo algebraičnosti
        \item Dokaži obstoj minimalnega polinoma algebraičnega elementa $a$
        \item Dokaži enoličnost minimalnega polinoma algebraičnega elementa $a$
        \item Navedi ekvavilence temu, da je $p(x)$ minimalni polinom algebraičnega elementa $a$
        \begin{itemize}
            \item 1. ekvavilenca
            \item 2. ekvavilenca
            \item 3. ekvavilenca
        \end{itemize}
        \item Kateri elementi so algebraični stopnje 1? Kaj je njihov minimalni polinom?
        \item Pokaži, da je vsako kompleksno število algebraično nad realnimi števili. Kaj je njegova stopnja algebraičnosti?
        \item Najdi transcedentni element v polju racionalnih funkcij. Zakaj ni algebraičen?
        \item Kdaj algebraičnim oz. transcedentnim elementom pravimo števila?
        \item Definiraj algebraično število
        \item Kako iz polinoma v $Q[X]$ dobimo polinom v $Z[X]$?
        \item Ali je $i$ algebraičen? Če ja, katere stopnje in katerega polinoma?
        \item Ali je praštevilo $p$ algebraično število? Če ja, katere stopnje in katerega polinoma?
        \item Kakšna je množica algebraičnih števil? Zakaj?
        \item Kakšna je množica transcedentnih števil?
    \end{enumerate}

    \subsection{Končne razširitve}
    \begin{enumerate}
        \item Kako lahko obravnavamo razširitev E polja F? Pokaži zakaj
        \item Definiraj končno razširitev E polja F
        \item Definiraj stopnjo razširitve. Kako jo označimo?
        \item Kako označujemo stopnjo razširitve v linearni algebri?
        \item Navedi primer končne razširitve. Kaj je baza razširitve kot vektorskega prostora? Kolikšna je stopnja razširitve?
        \item Navedi primer, ki ni končna razširitev $Q$. Zakaj ni?
        \item IZREK - O "tranzitivnosti" končnih razširitev. Dokaži
        \begin{itemize}
            \item Definiraj baze
            \item Pokaži, da je "baza" E nad F ogrodje
            \item Pokaži, da je "baza" E nad F linearno neodvisna
        \end{itemize}
        \item POSLEDICA - Kdaj $[L:F]$ deli $[E:F]$. Dokaži
        \item Definiraj algebraično razširitev
        \item Definiraj transcedentno razširitev
        \item TRDITEV - Katere razširitve so algebraične? Dokaži
        \item Ali velja obrat trditve?
        \item Kaj označuje $F[A]$?
        \item Kaj označuje $F(A)$?
        \item Kako pišemo ti dve množici, če je A končna?
        \item Eksplicitno zapiši ti dve množici v primeru, ko $A = {a}$
        \item Kako imenujemo $F(A)$?
        \item Definiraj enostavno razširitev polja. Kako imenujemo element $a$?
        \item PRIMER - Kakšne oblike so elementi $f(i)$ za $f(x) \in Q[X]$? Zakaj?
        \item Čemu je enak $R[X]$? Kaj zato velja?
        \item IZREK - V skladu z novo vpeljanimi oznakami povej kaj velja za algenraičen element stopnje $n$ $a$
        \begin{itemize}
            \item Pokaži, da je $F[A]$ polje (vsebovanost inverznih elementov)
            \item Dokaži enakost $F[A]$ in množice, ki si jo definirala (vzami nek element iz $F[A]$ in pošiči njegovo stopnjo)
            \item Pokaži stopnjo algebraičnosti (poišči bazo)
        \end{itemize}
        \item PRIMER - Poišči bazo $Q(n-root{p})$
        \item PRIMER - Kaj je $Q(\sqrt{2})$
        \item PRIMER - Kaj je $Q(tretji koren iz 2)$
        \item OPOMBA - Poišči epimorfizem kolobarjev $F[X] \rightarrow F[a]$
        \begin{itemize}
            \item Poišči izomorfizem, če je $a$ algebraičen
            \item Poišči izomorfizem, če je $a$ transcedenten
        \end{itemize}
        \item Kaj velja za stopnjo algebraičonsti polja, $L$, ki vsebuje $F$?
        \item IZREK - Razširi prejšni izrek na višje dimenzije
        \begin{itemize}
            \item Kaj sledi iz indukcijske predpostavke?
            \item Kako iz tega dobiš končno razširitev polja F z n algebraičnimi elementi?
            \item Zakaj je kolobar $F[a_1, ... a_n]$ polje?
        \end{itemize}
        \item Opiši polje $Q(\sqrt{2}, \sqrt{3})$. Kolikšna je njegova stopnja algebraičnosti nad poljem $Q$?
        \item Navedi izrek o primitivnem elementu
        \item PRIMER - Čemu je enak $Q(\sqrt{2}, \sqrt{3})$?
        \item POSLEDICA - Algebraično opiši množico vseh elementov iz $E$, ki so algebraični nad $F$. Dokaži
        \item Navedi primer podpolja algebraičnih števil
        \item Navedi primer polja algebraičnih števil nad nekim poljem, ki je algebraična razširitev, a ni končna
    \end{enumerate}

    \subsection{Konstrukcije z ravnilom in šestilom}
    \begin{enumerate}
        \item Poimenuj tri probleme antične grčije
        \item Opiši podvojitev kocke in formaliziraj
        \item Opiši trisekcijo kota in formaliziraj
        \item Opiši kvadraturo kvadrata in formaliziraj
        \item Kako iz dane množice točk konstruiramo novo točko z ravnilom in šestilom?
        \item Definiraj konstruktabilne točke
        \item Kaj je največja množica konstruktabilnih točk?
        \item Definiraj konstruktabilna števila
        \item Zapiši lemo, s katero izpeljemo glavni izrek tega poglavja. Dokaži (razdeli na tri dele)
        \item Navedi glavni izrek tega poglavja. Koliko je stopnja algebraičnosti komponent v izreku?
        \item Razloži zakaj ni možno - podvojitev kocke
        \item Razloži zakaj ni možno - trisekcija kota
        \item Razloži zakaj ni možno - kvadratura kroga
    \end{enumerate}

    \subsection{Kratnost ničle polinoma}
    \begin{enumerate}
        \item TRDITEV - Natanko kdaj ima polinom ničlo $a \in E$? Pazi od kod jemlješ polinome!
        \item Definiraj enostavno ničlo
        \item Definiraj $k$-kratno ničlo
        \item PRIMER - Najdi in kategoriziraj ničle $x^4 - 2x^3 + 2x - 1$
        \item Kako pridemo do polinoma brez ničel v $E$?
        \item Koliko ničel ima $f$?
        \item Pokaži, da $f$ nima drugih ničel kot teh, ki jih dobimo v razcepu
        \item Kolikšne stopnje je $f$?
        \item TRDITEV - Največ koliko ničel ima neničeln polinom?
        \item PRIMER - Dan imaš polinom $f(x) = x^6 - 3x^4 + 4$ Najdi število ničel in polinom brez ničel v $Q$, $R$ in $C$
        \item Kdaj so ničle polinomov enostavne?
        \item Definiraj odvod polinoma
        \item Kolikšne stopnje je odvod polinoma? Kdaj ta enačba velja? Zakaj drugače ne velja?
        \item IZREK - Kdaj so vse ničle polinoma enostavne? Dokaži
        \item Definiraj seperabilen polinom 
        \item Definiraj seperabilno razširitev
        \item Definiraj perfektno polje
        \item Navedi primera perfektnih polj
        \item Navedi primer polja, ki ni perfektno. Koliko je njegova karakteristika?
    \end{enumerate}
\end{document}