\documentclass{article}
\usepackage{amsmath}
\usepackage{xcolor}
\title{Afina in projektivna geometrija}

\begin{document}
    \maketitle

    \section{Uvod}
    \begin{enumerate}
        \item Kaj je $n$-razsežna geometrija?
        \item V kakšni relaciji so elementi množic? Elementi katerih množic so v relaciji?
        \item Naštej 3 geometrije
        \item Definiraj transformacijo. 
        \item Kaj pomeni, da so transformacije usklajene?
        \item Kaj ohranjajo transformacije Evklidske ravnine?
        \item Navedi razlog o aksiomu, zakaj je dobro vpeljat neevklidske geometrije
        \item Katera geometrija se je potem vpeljala?
        \item Navedi drugi in tretji programerski razlog za vpeljavo neevklidske geometrije
        \item Navedi četrti lepotni razlog za vpeljavo neevklidske geometrije
        \item Katera geometrija se zaradi tega vpelje?
    \end{enumerate}

    \section*{AFINA GEOMETRIJA}
    \section{Afini podprostori v vektorskem prostoru}
    \begin{enumerate}
        \item Definiraj afin podprostor
        \item Kaj pomeni beseda afin?
        \item Definiraj afin prostor
        \item LEMA - Kako zamenjamo element okoli katerega naredimo afin prostor? Dokaži.
        \item POSLEDICA - V kakšnem velikostnem razmerju sta dva vektorska prostora afinih prostorov? Dokaži.
        \item POSLEDICA - Kdaj sta vektorska prostora afinih prostoroc enaka? Dokaži.
        \item Definiraj razsežnost afinega prostora
        \item Kaj so 1-razsežni afini prostori? Kaj so dvorazsežni?
        \item V $Z_2^2$ najdi 6 afinih premic
        \item Definiraj afino kombinacijo
        \item TRDITEV - O afini kombinaciji. Nad kakšnim obsegom mora biti afin prostor definiran?
        \item Zakaj obseg ne sme imeti karakteristike 2?
        \item LEMA - Navedi lemo, s katero razširimo trditev
        \item Dokaz leme
        \begin{itemize}
            \item S katerim matematičnim načelom dokazujemo?
            \item Razdeli primer $n = 3$ na dva dela
            \item Dokaži za $n >= 4$, če vsota prvih $n - 1$ elementov ni 0
            \item Dokaži za $n >= 4$, če vsota prvih $n - 1$ elementov 0
        \end{itemize}
        \item Dokaz trditve
        \begin{itemize}
            \item Dokaži da: A afin -> lin. kombinacija je v A
            \item Definiraj tak vektorski prostor, da bo zanj A afin prostor
            \item Pokaži, da je res vektorski podprostor afinega prostora
        \end{itemize}
        \item TRDITEV - Kaj sledi iz leme in trditve?
        \item TRDITEV - Kaj velja za presek družine afinih podprostorov. Dokaži
        \item Ali velja kaj podobnega za unije? Zakaj?
        \item Definiraj afino ogrinjačo
        \begin{itemize}
            \item Kaj je oznaka?
            \item Ali je sama afin prostor? Zakaj?
        \end{itemize}
        \item TRDITEV - Čemu je afina ogrinjača enaka? Dokaži obojestransko vsebovanost
        \item Od česa je odvisno, če se afina podprostora sekata? Nariši skico, ko se ne
        \item TRDITEV - Kdaj se afina prostora sekata? Dokaži
        \item Kaj bi veljalo, če bi bil ta vektor vsebovan v direktnem produktu podprostorov $W$ in $U$?
        \item LEMA - Formalno zapiši: Afina ogrinjača vsebuje A, B in morda tudi premico, ki povezuje a in b
        \begin{itemize}
            \item Definiraj prostor $T$ in z njim dokaži $\subseteq$
            \item Definiraj prostor $C$ in z njim dokaži $\supseteq$ 
        \end{itemize}
        \item TRDITEV - Kaj je dimenzija unije dveh disjunktnih afinih prostorov? Dokaži
        \item Defniraj kdaj sta dva afina prostora vzporedna
        \begin{itemize}
            \item Oznaka
        \end{itemize}
        \item IZREK - Natanko kdaj sta $\mathcal{A}$ in $\mathcal{B}$ vzporedna, če njun presek ni prazen? Dokaz
        \item IZREK - Natanko kdaj sta $\mathcal{A}$ in $\mathcal{B}$ vzporedna, če je njun presek prazen? Dokaz
        \item Definiraj afino neodvisnost
        \item Definiraj afino bazo
        \item IZREK - Kako iz bate za $A$ dobimo bazo za $W$?
        \item IZREK - Kako iz bate za $W$ dobimo bazo za $A$?
        \item POSLEDICA - Dopolnitev do baze. Dokaži
    \end{enumerate}

    \section{Semilinearne preslikave}
    \begin{enumerate}
        \item Definiraj semilinearno preslikavo
        \item Navedi semilinearno preslikavo in njen njen avtomorfizem nad obsegom $C$
        \item Kaj je grupa avtomorfizmov obsegov $R$, $Q$, $Z_p$? Kaj so zato njihove semilinearne preslikave?
        \item S čem je semilinearna preslikava enolično določena?
        \item TRDITEV - Semilinearna preslikava slika podprostore v podprostore. Dokaži
        \item TRDITEV - Inverz semilinearne preslikave slika podprostore v podprostore. Dokaži
        \item POSLEDICA - Kateri dvve algebraični strukturi sta zato vektorska posprostora?
        \item TRDITEV - Kakšna preslikava slika linearno neodvisne vektorje v linearno neodvisne? Dokaži
        \item Dokaži: Kompozitum semilinearnih preslikav je semilinearna preslikava
        \item Dokaži: Kompozitum semilinearne preslikave je semilinearna preslikava
        \item Kaj sledi iz teh dveh trditev?
        \item Kdaj je vsota semilinearnih preslikav semilinearna preslikava?
    \end{enumerate}

    \section{Afine transformacije}
    \begin{enumerate}
        \item Definiraj afino transformacijo
        \item IZREK - O premicah
        \item OPOMBA - Zakaj izrek ne sledi direktno iz definicije?
        \item Kdaj je slika premice vedno kar cela premica? Povej to isto stvar z inj, surj, bij
        \item LEMA - Navedi pogoje in formaliziraj, da poljubna točka iz afine ogrinjače unije dveh afinih prostorov leži na premici
        \item Zakaj je potrebno privzeti, da obseg ni $Z_2$?
        \item Zakaj je potrebno privzeti, da presek ni prazen?
        \item Dokaz leme
        \begin{itemize}
            \item Definiraj bazo preseka in dopolni do baz A in B
            \item Zapiši točko c s pomočjo baznih vektorjev
            \item Definiraj $\alpha$ in $\beta$
            \item Pokaži, ko sta $\alpha$ in $\beta$ različna od 0
            \item Pokaži, ko en od $\alpha$ oziroma $\beta$ enak 0
        \end{itemize}
        \item LEMA - Tri trditve, ki veljajo za dve premici, kjer je slika ene vsebovana v drugi
        \item Dokaz leme
        \begin{itemize}
            \item Pravilno izberi $a$
            \item Definiraj $\mathcal{A}_2$. Koliko je dimenzija?
            \item Pokaži, da je $\tau (\mathcal{A}_2)$ vsebovan v $q$
            \item Kaj iz tega sledi za dimenzijo slike $\mathcal{A}_2$?
            \item Koliko je dimenzija $\mathcal{A}_i$?
            \item Koliko je dimenzija $\tau (\mathcal{A}_i)$?
        \end{itemize}
        \item Dokaži izrek o premici
        \item Kam afina transformacija slika nakolinearne točke? Dokaži
        \item Dokaži, da je množica afinih transformacij grupa.
        \item Kam afina transformacija preslika koplanarne točke?
        \begin{itemize}
            \item Dokaži, če so tri kolinearne
            \item Dokaži, če so po tri nekolinearne
        \end{itemize}
        \item Kam afina transformacija slika vzporedne premice? Dokaži
    \end{enumerate}

    \section{Osnovni izrek afine geometrije}
    \begin{enumerate}
        \item Navedi osnovni izrek aritmetike
        \item Dokaz
        \begin{itemize}
            \item ( <= ) Kam se slika afina premica?
            \item ( => ) Definiraj preslikavo A. Kompozitum katerih preslikav je?
            \item ( => ) Dokaži njeno aditivnost
            \item ( => ) Kako dosežemo linearno neodvisnost, če sta dani točki x in y odvisni?
            \item ( => ) Dokaži njeno semihomogenost
            \item ( => ) Poišči avtomorfizem in dokaži, da je neodvisen od izbire vektorja. Kaj pa če sta x in y linearno odvisna?
            \item ( => ) Pokaži, da je avtomorfizem
        \end{itemize}
        \item Kaj velja za afino transformacijo iz izreka?
        \item Definiraj dilatacijo
        \item IZREK - Kaj ohranja dilatacija? Dokaži
        \item Navedi osnovni izrek afine geometrije za dilatacije
        \begin{itemize}
            \item ( <= ) Pokaži, da sta afin podpostor in njegova slika vzporedna
            \item ( => ) Kakšen predpis takoj velja?
            \item ( => ) Definiraj premico skozi 0 in x, prestavi jo. Kam se slika translirana premica?
            \item ( => ) Kako dobimo skalar?
            \item ( => ) Pokaži, da je skalar neodvisen od izbire vektorja
        \end{itemize}
        \item Definiraj translacijo
        \item Kdaj je dilatacija translacija? Dokaži
        \item Definiraj afino geometrijo
        \item Definiraj kdaj sta afini geometriji izomorfni
        \item Kaj pomeni, da preslikava ohranja inkluzije?
        \item Kdaj sta afini geometriji izomorfni, če sta afina prostora definirana nad istim obsegom?
    \end{enumerate}

    \section{Aksiomatsko definirana afina ravnina}
    \begin{enumerate}
        \item Definiraj aksiomatsko definirano afino ravnino
        \item Kdaj sta premici vzporedni?
        \item Katerim aksiomom zadošča aksiomatsko definirana afina ravnina (3)?
        \item Katerih razsežnosti $O^n$ so in katerih niso aksiomtsko definirane afine ravnine? Zakaj ni?
        \item Definiraj Moultonovo ravnino
        \item Navedi prvi Desarguesov izrek
        \item Ali prvi Desarguesov izrek velja v Moultonovi ravnini? Zakaj ja/ne?
        \item Definiraj afino transformacijo
        \item Definiraj dilatacijo
        \item Definiraj translacijo
        \item Kaj velja za afine transformacije?
        \item Ali obstaja afina transformacija iz Moultonove ravnine?
        \item Čemu je ekvavilenten prvi Desarguesov izrek?
        \begin{itemize}
            \item Dokaži (A4 => prvi Desarguesov) 
            \item Izberi potrebne točke in premice
            \item Uporabi A4
            \item Kaj bi veljalo, če $\tau (Q) \neq Q'$
            \item Dokončaj dokaz
            \item Dokaži (prvi Desarguesov => A4)
            \item Konstruiraj $\tau$
            \item Preveri, da je dobro definirana
        \end{itemize}
        \item Definiraj usmerjeno daljico
        \item Kdaj sta usmerjeni daljici ekvavilentni?
        \item Definiraj vektor
        \item Navedi drugi Desarguesov izrek
        \item Kateremu aksiomu A5 je ekvavilenten izrek?
        \item Kaj omogoča aksiom A4?
        \item Kaj omogoča aksiom A5?
        \item IZREK - Kaj velja za afin prostor, ki zadošča aksiomom A1 do A5?
        \item Navedi Papppusov izrek
        \item Nad kakšnim obsegom velja Pappusov izrek?
    \end{enumerate}


    \section{PROJEKTIVNA GEOMETRIJA}
    \begin{enumerate}
        \item Kako iz afine ravnine dobimo projektivno?
        \item Kako zgleda vložitev afine $Z^2_2$ v projektivno?
        \item Kaj dobimo, če hiperboli dodamo točke v neskončnosti?
        \item Zapiši predpis s katerim preslikamo hiperbolo
        \item Zapiši predpis s katerim preslikamo parabolo
        \item Kakšen mora biti obseg nad katerim delamo? Zakaj?
        \item Kam vložimo realno afino ravnino $A$?
        \item Kaj so točke v projektivni geometriji?
        \item Kaj so premice v projektivni ravnini?
        \item Kaj je presek dveh afinih premic v projektivnem? (izpelji)
        \item Definiraj projektivno geometrijo
        \begin{itemize}
            \item Oznaka?
        \end{itemize}
        \item Kaj so v projektivnem 1-razsežni podprostori, 2-razsežni podprostori?
        \item Definiraj projektivno razsežnost?
        \item Kolikšna je razsežnost geometrije $P(V)$?
        \item Kako iz projektivnih točk konstruiramo projektivno premico?
        \item Kaj je dimenzija preseka projektivnih premic v projektivni geometriji projektivne razsežnosti 2?
        \item Kje se sekata različni projektivni premici?
    \end{enumerate}

    \section{Dualnost}
    \begin{enumerate}
        \item Definiraj dualni vektorski prostor
        \begin{itemize}
            \item Oznaka?
            \item Zakaj rabimo polje in ne obseg?
        \end{itemize}
        \item Kaj je dualna baza baze $V$?
        \item Kaj sta si torej $V$ in $V*$?
        \item Kaj pa $V$ in $V**$?
        \item Poišči izomorfizem med $V$ in $V**$
        \item Definiraj zgornji anhilator
        \item Definiraj spodnji anhilator
        \item Kaj sta anhilatorja?
        \item Definiraj bijekcijo med $P(V)$ in $P(V*)$ in njen inverz
        \item Dokaži, da sta si inverzni (v dveh točkah)
        \item IZREK - Uporabi zgornji anhilator na $W_1 <= W_2 <= V$. Dokaži
        \item IZREK - Čemu je enak zgornji angilator $(W_1  \cap W_2)$? Dokaži obojestransko vsebovanost
        \item IZREK - Čemu je enak zgornji angilator $(W_1  + W_2)$? Dokaži obojestransko vsebovanost
        \item IZREK - Kaj je kodimenzija $V$? Dokaži
        \item Kaj je izjava o geometriji $P(V)$?
        \item Kaj je dualna trditev?
        \item Kaj je dualna izjava k izjavi: V projektivni ravnini skozi poljubni točki poteka natanko ena premica
        \begin{itemize}
            \item Zapiši izjavo v formalni obliki
            \item Zapiši njej dualno izjavo
        \end{itemize}
        \item Kaj je dualna izjava k izjavi: $A, B, C$ so kolinearne točkev projektivni ravnini $P(V)$
        \begin{itemize}
            \item Zapiši izjavo v formalni obliki
            \item Zapiši njej dualno izjavo
        \end{itemize}
        \item IZREK - Navedi PRINCIP DUALNOSTI
        \item Definiraj trikotnik
        \item Definiraj kdaj sta trikotnika $ABC$ in $A'B'C'$ v perspektivni legi
        \item Navedi DESARGUESOV IZREK v projektivni geometriji
        \item Dokaz
        \begin{itemize}
            \item (=>) Izberi 7 neničelnih vektorjev
            \item Kako bi z njimi zapisala presečišče $AA'$, $BB'$ in $CC'$?
            \item Kako bi z njimi zapisala točke, ki so presečišča stranic?
            \item Kaj moramo dokazati za te točke?
            \item Kaj je dualna izjava k tej izjavi?
            \item Kako dokažemo nasprotno implikacijo? 
        \end{itemize} 
    \end{enumerate}

    \section{Vložitev afine geometrije v projektivno}
    \begin{enumerate}
        \item Definiraj kake dimenzije naj bo $V$, $W <= V$ in afin podprostor prostora $V$
        \item Kako afinemu prostoru priredimo projektivni podprostor? Definiraj preslikavo
        \item LEMA - Koliko je $l(x + U)$? Dokaži
        \item LEMA - Koliko je $l(x + U) \cap A$?
        \item LEMA - Formalno zapiši, da vsak $Z$, ki ni cel vsebovan v $W$, je slika nečesa
        \begin{itemize}
            \item Pokaži, da je presek neprazen
            \item Pokaži, da je $Z$ slika
        \end{itemize}
        \item Naštej 7 lastnosti, ki veljajo za preslikavo $l$
        \item Dokaži, da je l injektivna
        \item Kaj je slika l aka kdaj je $Z$ v zalogi vrednosti $l$? Dokaži
        \item Dokaži, da $l$ ohranja inkluzije
        \item Kaj velja za družino afinih podprostorov z nepraznim presekom? Dokaži
        \item Kaj velja za družino afinih podprostorov? Dokaži
        \item Kaj je dimenzija podprostora afine ravnine? Dokaži
        \item Kdaj sta afina podprostora vzporedna? Dokaži
        \item Kaj je $l$?
        \item Kako označimo vse točke v $P(V)$?
        \item Kaj sledi iz vzporednosti p in q za projektivno ravnino?
        \item Kaj sledi iz vzporednosti p, q in r za projektivno ravnino? Kako to imenujemo?
        \item Navedi drugi Desarguesov izrek za afino ravnino
        \item Dokaz
        \begin{itemize}
            \item Kdaj moramo vse translirati?
            \item Definiraj vložitev afine v projektivno
            \item Kaj sledi iz ohranjana inkluzij?
            \item Kaj velja za trikotnika v projektivnem?
            \item Kaj velja po Desarguesovem izreku za projektivno ravnino?
            \item Kaj torej sledi (točke v neskončnosti)
        \end{itemize}
        \item Navedi prvi Desarguesov izrek za afino ravnino. Začni dokaz
    \end{enumerate} 

    \section{Kolineacije in projektivnosti}
    \begin{enumerate}
        \item Definiraj kolineacijo
        \item PRIMER - Definiraj kolineacijo za bijektivno semilinearno preslikavo in pokaži, da je kolineacija
        \item Kaj velja če $L < L_1 + ... L_k$, kjer so $L, L_i \in PV$. Dokaži z indukcijo
        \item Do česa lahko razširimo kolinealizacijo? Definiraj njen predpis
        \item Kaj je kolineacija vsote projektivnih podprostorov? Dokaz
        \item Kaj je kolineacija preseka projektivnih podprostorov? Dokaz - poglej kako so povezane dimenzije vsot in preseka + uporabi ohranjanje inkluzij
        \item Navedi osnovni izrek projektivne geometrije (dokaz)
        \item Kaj je kolineacija porojena s semilinearno preslikavo?
        \item Definiraj projektivnost
        \item V katerih primerih je kolineacija kar projektivnost?
        \item Kakšen algebraičen objekt je množica vseh kolineacij? Za katero operacijo?
        \item Definiraj grupo obrnljivih linearnih preslikav
        \item Definiraj grupo obrnljivih semilinearnih preslikav
        \item Definiraj grupo projektivnosti
        \item Definiraj grupo kolineacij
        \item Od kod kam slikajo preslikave v teh grupah?
        \item Kateri grupi sta projektivni?
        \item Kaj enačimo z $GL(v)$ in $\Gamma L(v)$?
        \item Navedi dva izomorfizma med definiranimi grupami. Dokaži, da sta res izomorfizma
        \item Definiraj projektivno ogrodje
        \item Kaj je projektivno ogrodje, če je $dimV = 2$? Zakaj?
        \item Kaj je projektivno ogrodje, če je $dimV = 2$? 
        \item TRDITEV - Kaj velja za projektivnost, ki vsako točko iz projektivnega ogrodja preslika vase? Dokaži
        \item POSLEDICA - Poišči projektivnost, ki slika projektivno ogrodje $PV$ v projektivno ogrodje $PV'$. Dokaži
        \item Zakaj je projektivnost $PR^2 -> PR^2$ določena s tremi točkami? 
    \end{enumerate}

    \section{Perspektivnost}
    \begin{enumerate}
        \item Definiraj perspektivnost s centrom T
        \item Navedi lemo o distributivnosti in jo dokaži
        \item Koliko je dimenzija slike prostora z dimenzijo 1?
        \item Čemu je enaka direktna vsota $X$ in $T$? Dokaži
        \item Kaj je inverz perspektivnosti s ventrom T? Dokaži
        \item Kaj je vsaka perspektivnost? Dokaz
    \end{enumerate}

    \section{Homogene koordinate}
    \begin{enumerate}
        \item Definiraj ekvavilenčno relacijo na $O^n \backslash {0}$ 
        \item Kako označimo ekvavilenčni razred?
        \item Kako točki $L$ priredimo homogene koordinate glede na projektivo ogrodje?
        \item Pokaži, da so homogene koordinate dobro definirane
        \item Pokaži, da so homogene koordinate odvisne le od izbire projektivnega ogrodja in ne od izbire vektorjev $u_i$
        \item Kako prehajamo med homogenimi in nehomogenimi koordinatami?
    \end{enumerate}

    \section{Dvorazmerje}
    \begin{enumerate}
        \item Kaj je motivacija, da definiramo dvorazmerje?
        \item Definiraj Dvorazmerje
        \item TRDITEV - Katere zveze obstajajo med dvorazmerji glede na različna ogrodja? Dokaži
        \item IZREK - Katera preslikava ohranja dvorazmerje? Dokaži
        \item Definiraj dvorazmerje šopa
        \item Kaj je dvorazmerje šopa v projektivni?
        \item Kaj pa če bi izbrali drugo premico $t$?
        \item Definiraj harmonično četverko
        \item Kaj je dvorazmerje realnih števil x, y, u, v?
        \item Kaj velja za dvorazmerje, če je $C$ točka v neskončnosti?
        \item IZREK - Kako konstruiramo harmonično četverko? Dokaži s pomočjo prejšnega izreka
        \item TRDITEV - Kdaj je projektivnost, ki ni identiteta, involucija?
    \end{enumerate}

    \section{Stožnice}
    \begin{enumerate}
        \item Definiraj algebraično množico
        \item Kaj je stožnica?
        \item Definiraj kdaj je polinom homogen?
        \item Kaj so homogeni polinomi za $d = 0$?
        \item Kaj so homogeni polinomi za $d = 1$?
        \item Kaj so homogeni polinomi za $d = 2$ in $n=0$?
        \item Ali homogen polinom definira preslikavo $PV -> O$?
        \item Definiraj projektivno algebraično množico
        \item Kaj je stožnica?
        \item Kako lahko podamo stožnico v projektivnem?
        \item Definiraj kvadratno formo in matriko, ki pripadata stožnici
        \item Definiraj simetrično bilinearno formo
        \item Kako dobimo simetrično bilinearno formo iz kvadratne forme?
        
        \item Kdaj sta si kvadratni formi ekvavilentni?
        \item Kdaj sta si stožnici ekvavilentni?
        \item Kako sta ti dve definiciji povezani? Dokaži
        \item Ali trditev velja v drugo smer?
        \item SYLVESTROV IZREK - Čemu je ekvavilentna forma na $R$ in čemu na $C$?
    \end{enumerate}

    \section{Klasifikacija stožnic v R in C}
    \begin{enumerate}
        \item Definiraj neizrojeno stožnico
        \item Glede na rang matrike $M$ določi kvadratno dormo in stožnico
        \item Katera stožnica je neizrojena?
        \item Kaj je signatura?
        \item Glede na signaturo določi kvadratno formo in stožnico
        \item V katerem primeru dobimo neizrojeno neprazno stožnico?
        \item IZREK - Koliko točk je v preseku neizrojene stožnice in projektivne premice?
        \item IZREK - Kdaj je presek neprazen?
        \item Kdaj je presek stožnice in projektivne premice prazen? (v realnem)
        \item Kdaj je presek stožnice in projektivne premice ena točka? (v realnem)
        \item Kdaj je presek stožnice in projektivne premice dve točki? (v realnem)
    \end{enumerate}

    \section{Pol in polara}
    \begin{enumerate}
        \item Definiraj tangento
        \item Ali je ta definicija dobra v evklidski ravnini?
        \item Definiraj polaro
        \item TRDITEV - Kdaj je polara projektivna premica? Dokaži
        \item TRDITEV - Kdaj je polara tangenta na $S$ v $A$?
        \item TRDITEV - Ali v vsaki točki obstaja tangenta?
        \item Pokaži, da je $A \in p_B \Leftrightarrow B in b_A$
        \item Pokaži, da če $A \neq B$, potem tudi njuni polari nista enaki
        \item Zapiši predpis za bijekcijo med točkami in premicami v projektivni ravnini. Pokaži, da je bijekcija
        \item Definiraj pol premice $p$ glede na neizrojeno stožnico $S$
        \item Kako konstruiramo polaro? (3 načini)
        \begin{itemize}
            \item Za točke na stožnici
            \item Za točke, za katere obstajajo tangente na stožnico
            \item Za točke za katere ne obstajajo tangente na stožnico 
        \end{itemize}
    \end{enumerate}

    \section{Dvorazmerja na stožnici}
    \begin{enumerate}
        \item TRDITEV - Katere štiri točke, ki jih tvorimo iz $A$ in premic, tvorijo harmonično četverko?
        \item Dokaz
        \begin{itemize}
            \item Definiraj harmonično četverko
            \item Kaj sledi iz tega, da je $B$ na polari $A$?
            \item Zakaj $\Phi(d, c) \neq 0$?
        \end{itemize}
        \item Navedi SETINERJEV IZREK
        \item OPOMBA - Kaj je polara, če je $T = A$?
        \item Dokaz
        \begin{itemize}
            \item Ali lahko ležijo točke $A, B, C in D$ na isti premici?
            \item Kaj tvorijo ${D, A, B, C}$?
            \item Kako lahko zapišemo $d$ in kako $t$??
            \item Zapiši matriko na bazi ${a, c, d}$
            \item Izberi $d'$. Kaj je potem $\beta'$
            \item Kakšna so razmerja med $\alpha'. \beta' in \gamma'$?
            \item Kakšno enačbo dobimo, ker $D$ leži na stožnici?
            \item Kakšno enačbo dobimo, ker $T$ leži na stožnici?
            \item Kaj je dvorazmerje, če so točke $A, B C, D, T$ različne?
            \item Uvedi oznake za $C'$ in $D'$. 
            \item Zakaj so koordinate razlčne?
            \item Kako računamo dvorazmerje $A, B, C', D'$?
            \item Uporabi izpeljani enačbi (D in T na stožnici)
            \item Kaj je dvorazmerje, če je $A = T$?
            \item Uvedi kaj sta $D'$ in $A'$
            \item Kako zračunamo $p_A$?
            \item Kako izračunamo dvorazmerje $A', B, C, D'$
        \end{itemize}
        \item Definiraj dvorazmerje točk na stožnici
        \item $A, B, C$ so dane točke na stožnici. Kako konstruiramo točko $D$ na stožnici, da $A, B, C, D$ tvorijo harmonično četverko?
        \item Navedi PASCALOV IZREK
        \item Dokaz (najprej sekaj dvorazmerje $D(A, B, C, B')$ z $A'$ in $E$, potem pa še z $C'$ in $E$)
        \item Ali velja obrat?
        \item Navedi BRAINCHONOV IZREK
        \item Dokaz
        \begin{itemize}
            \item Označi dotikališča
            \item Kaj so njihove polare?
            \item Kaj so polare točk v izreku?
            \item Kaj označimo z $X, Y, Z$?
            \item Kaj so polare teh točk?
            \item Dokoči dve točki (presečišči) kot v izreku in pokaži, da sta ista točka
        \end{itemize}
    \end{enumerate}
\end{document}
